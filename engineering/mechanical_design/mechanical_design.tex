\documentclass[11pt]{article}
\usepackage[T1]{fontenc}
\usepackage{lmodern}
\usepackage[margin=1in]{geometry}
\usepackage{amsmath,amssymb}
\usepackage{array}
\usepackage{booktabs}

\pagestyle{empty}
\setlength{\parindent}{0pt}
\setlength{\parskip}{8pt}

\begin{document}

\begin{center}
\LARGE{\textbf{Mechanical Design}}
\end{center}

\vspace{10pt}

\section*{Design Philosophy and Safety}

\subsection*{Factor of Safety}

\textbf{Based on Strength:}
$$n = \frac{S_{\text{strength}}}{S_{\text{stress}}}$$

\textbf{Based on Load:}
$$n = \frac{P_{\text{failure}}}{P_{\text{applied}}}$$

Typical values:
\begin{itemize}
\item Known materials, certain loads: $n = 1.5$ to $2.5$
\item Uncertain loads or materials: $n = 3$ to $4$
\item Life-critical applications: $n > 4$
\end{itemize}

\subsection*{Design Process}

1. Define requirements and constraints
2. Perform preliminary analysis
3. Select materials and components
4. Detailed stress analysis
5. Check for failure modes
6. Iterate and optimize

\section*{Static Failure Theories}

\subsection*{Ductile Materials}

\textbf{Maximum Shear Stress Theory (Tresca):}
$$\frac{\sigma_1 - \sigma_3}{2} \leq \frac{S_y}{n}$$

or
$$\sigma_1 - \sigma_3 \leq \frac{S_y}{n}$$

\textbf{Distortion Energy Theory (von Mises):}

For 3D stress state:
$$\sigma' = \sqrt{\frac{1}{2}[(\sigma_1 - \sigma_2)^2 + (\sigma_2 - \sigma_3)^2 + (\sigma_3 - \sigma_1)^2]} \leq \frac{S_y}{n}$$

For plane stress ($\sigma_3 = 0$):
$$\sigma' = \sqrt{\sigma_1^2 - \sigma_1\sigma_2 + \sigma_2^2} \leq \frac{S_y}{n}$$

For simple stress states:
$$\sigma' = \sqrt{\sigma^2 + 3\tau^2} \leq \frac{S_y}{n}$$

\textbf{Note:} von Mises theory generally more accurate for ductile materials

\subsection*{Brittle Materials}

\textbf{Maximum Normal Stress Theory (Rankine):}
$$|\sigma_1| \leq \frac{S_{ut}}{n} \quad \text{or} \quad |\sigma_3| \leq \frac{S_{uc}}{n}$$

where $S_{ut}$ is ultimate tensile strength and $S_{uc}$ is ultimate compressive strength

\textbf{Modified Mohr Theory:}

Accounts for different tensile and compressive strengths

Most conservative and commonly used for brittle materials

\section*{Fatigue Failure}

\subsection*{S-N Curve (Stress-Life)}

\textbf{Endurance Limit:}

For steel: $S_e' \approx 0.5 S_{ut}$ (up to $S_{ut} = 1400$ MPa or 200 ksi)

For non-ferrous materials: No true endurance limit; use strength at $N = 5 \times 10^8$ cycles

\textbf{Modified Endurance Limit:}
$$S_e = k_a k_b k_c k_d k_e k_f S_e'$$

where:
- $k_a$ = surface finish factor
- $k_b$ = size factor
- $k_c$ = load factor
- $k_d$ = temperature factor
- $k_e$ = reliability factor
- $k_f$ = miscellaneous effects factor

\textbf{Surface Finish Factor:}
$$k_a = a S_{ut}^b$$

\begin{center}
\begin{tabular}{lcc}
\toprule
\textbf{Surface Finish} & $a$ & $b$ \\
\midrule
Ground & 1.34 & $-0.085$ \\
Machined & 4.51 & $-0.265$ \\
Cold-drawn & 4.51 & $-0.265$ \\
Hot-rolled & 57.7 & $-0.718$ \\
As-forged & 272 & $-0.995$ \\
\bottomrule
\end{tabular}
\end{center}

($S_{ut}$ in MPa for these values)

\textbf{Size Factor:}
$$k_b = \begin{cases} 
\left(\frac{d}{7.62}\right)^{-0.107} & 2.79 \leq d \leq 51 \text{ mm} \\
0.91d^{-0.157} & d > 51 \text{ mm}
\end{cases}$$

For non-rotating round: $d_{eq} = 0.808\sqrt{bh}$ (rectangular $b \times h$)

\textbf{Load Factor:}
- Bending: $k_c = 1$
- Axial: $k_c = 0.85$
- Torsion: $k_c = 0.59$

\textbf{Temperature Factor:}
$$k_d = \begin{cases}
1 & T \leq 450°C \\
1 - 0.0058(T - 450) & 450 < T < 550°C
\end{cases}$$

\textbf{Reliability Factor:}

\begin{center}
\begin{tabular}{cc}
\toprule
\textbf{Reliability} & $k_e$ \\
\midrule
50\% & 1.000 \\
90\% & 0.897 \\
95\% & 0.868 \\
99\% & 0.814 \\
99.9\% & 0.753 \\
\bottomrule
\end{tabular}
\end{center}

\subsection*{Fluctuating Stress}

\textbf{Mean and Alternating Stress:}
$$\sigma_m = \frac{\sigma_{max} + \sigma_{min}}{2}$$
$$\sigma_a = \frac{\sigma_{max} - \sigma_{min}}{2}$$

\textbf{Stress Ratio:}
$$R = \frac{\sigma_{min}}{\sigma_{max}}$$

\textbf{Completely Reversed:} $R = -1$ ($\sigma_m = 0$)

\textbf{Zero-to-Max:} $R = 0$ ($\sigma_m = \sigma_a$)

\subsection*{Fatigue Failure Criteria}

\textbf{Goodman Criterion (conservative):}
$$\frac{\sigma_a}{S_e} + \frac{\sigma_m}{S_{ut}} = \frac{1}{n}$$

\textbf{Gerber Criterion (less conservative):}
$$\frac{\sigma_a}{S_e} + \left(\frac{\sigma_m}{S_{ut}}\right)^2 = \frac{1}{n}$$

\textbf{Soderberg Criterion (most conservative):}
$$\frac{\sigma_a}{S_e} + \frac{\sigma_m}{S_y} = \frac{1}{n}$$

\textbf{ASME Elliptic (for shafts):}
$$\left(\frac{\sigma_a}{S_e/n}\right)^2 + \left(\frac{\sigma_m}{S_y/n}\right)^2 = 1$$

\subsection*{Combined Stresses in Fatigue}

\textbf{von Mises Approach:}

$$\sigma_a' = \sqrt{\sigma_a^2 + 3\tau_a^2}$$
$$\sigma_m' = \sqrt{\sigma_m^2 + 3\tau_m^2}$$

Then apply Goodman or other criterion with $\sigma_a'$ and $\sigma_m'$

\subsection*{Stress Concentration in Fatigue}

\textbf{Fatigue Stress Concentration Factor:}
$$K_f = 1 + q(K_t - 1)$$

where:
- $K_t$ = theoretical stress concentration factor
- $q$ = notch sensitivity ($0 \leq q \leq 1$)
- $q = 0$: no sensitivity (ignore $K_t$)
- $q = 1$: full sensitivity (use full $K_t$)

\textbf{Notch Sensitivity:}
$$q = \frac{1}{1 + a/\sqrt{r}}$$

where $a$ is Neuber constant (material property) and $r$ is notch radius

\textbf{Applying $K_f$:}

For alternating stress only:
$$\sigma_a = K_f \sigma_{a,nominal}$$

Mean stress usually not affected by $K_f$

\subsection*{Cumulative Damage (Miner's Rule)}

$$\sum_{i=1}^{k} \frac{n_i}{N_i} = C$$

where:
- $n_i$ = number of cycles at stress level $i$
- $N_i$ = cycles to failure at stress level $i$
- $C$ = damage sum (failure when $C \geq 1$)

Typically use $C = 1$ for design

\section*{Shaft Design}

\subsection*{ASME Code for Transmission Shafts}

\textbf{For ductile materials with yield strength:}

$$d = \left[\frac{16n}{\pi S_y}\sqrt{(K_tM_a + K_{tm}M_m)^2 + \frac{3}{4}(K_{ts}T_a + T_m)^2}\right]^{1/3}$$

where:
- $d$ = shaft diameter
- $n$ = factor of safety
- $M_a$, $M_m$ = alternating and mean bending moments
- $T_a$, $T_m$ = alternating and mean torques
- $K_t$, $K_{tm}$, $K_{ts}$ = fatigue stress concentration factors

\textbf{Simplified for steady loading:}

For rotating shaft with steady torque:
$$d = \left[\frac{16n}{\pi S_y}\sqrt{4(K_tM)^2 + 3(K_{ts}T)^2}\right]^{1/3}$$

\textbf{For infinite life (fatigue):}

$$d = \left[\frac{16n}{\pi}\sqrt{\left(\frac{K_fM_a}{S_e}\right)^2 + \frac{3}{4}\left(\frac{K_{fs}T_a}{S_e}\right)^2}\right]^{1/3}$$

\subsection*{Critical Speed}

\textbf{First Critical Speed (simply supported):}
$$\omega_c = \sqrt{\frac{g}{\delta_{static}}}$$

where $\delta_{static}$ is static deflection at center

For design, operating speed should be:
$$\omega_{operating} < 0.8\omega_c \quad \text{or} \quad \omega_{operating} > 1.2\omega_c$$

\section*{Springs}

\subsection*{Helical Compression Springs}

\textbf{Shear Stress:}
$$\tau = K_s\frac{8FD}{\pi d^3}$$

where:
- $F$ = axial force
- $D$ = mean coil diameter
- $d$ = wire diameter
- $K_s$ = shear stress correction factor (Wahl factor)

\textbf{Wahl Correction Factor:}
$$K_s = \frac{4C - 1}{4C - 4} + \frac{0.615}{C}$$

where $C = D/d$ is spring index (typically $4 \leq C \leq 12$)

\textbf{Spring Rate (Stiffness):}
$$k = \frac{F}{\delta} = \frac{Gd^4}{8D^3N_a}$$

where:
- $G$ = shear modulus
- $N_a$ = number of active coils

\textbf{Deflection:}
$$\delta = \frac{8FD^3N_a}{Gd^4}$$

\textbf{Solid Height:}
$$L_s = d(N_t + 1)$$

where $N_t$ is total number of coils

\textbf{Free Length:}
$$L_f = L_s + \delta_{max} + (\text{clash allowance})$$

Clash allowance typically 10-15\% of maximum deflection

\subsection*{Spring Design Considerations}

\textbf{End Conditions:}

\begin{center}
\begin{tabular}{lcc}
\toprule
\textbf{End Type} & $N_t$ & \textbf{Solid Height} \\
\midrule
Plain & $N_a$ & $dN_t$ \\
Plain and ground & $N_a$ & $dN_t$ \\
Squared & $N_a + 2$ & $d(N_t - 1)$ \\
Squared and ground & $N_a + 2$ & $d(N_t - 1)$ \\
\bottomrule
\end{tabular}
\end{center}

\textbf{Buckling:}

For $L_f/D > 4$, check for buckling

\subsection*{Helical Extension Springs}

Similar formulas to compression springs, but:
- Initial tension $F_i$ often present
- Hooks/loops add stress concentration
- $K_s$ includes hook effects

\subsection*{Torsion Springs}

\textbf{Bending Stress:}
$$\sigma = K_b\frac{32M}{\pi d^3}$$

\textbf{Angular Deflection:}
$$\theta = \frac{64MDN_a}{Ed^4}$$

where $M$ is applied moment and $E$ is Young's modulus

\section*{Screws and Fasteners}

\subsection*{Power Screws}

\textbf{Torque to Raise Load:}
$$T_R = \frac{Fd_m}{2}\left(\frac{l + \pi\mu d_m}{\pi d_m - \mu l}\right)$$

\textbf{Torque to Lower Load:}
$$T_L = \frac{Fd_m}{2}\left(\frac{\pi\mu d_m - l}{\pi d_m + \mu l}\right)$$

where:
- $F$ = load
- $d_m$ = mean diameter
- $l$ = lead ($l = p$ for single thread, $l = np$ for $n$ threads)
- $p$ = pitch
- $\mu$ = coefficient of friction

\textbf{Efficiency:}
$$e = \frac{Fl}{2\pi T_R}$$

\textbf{Self-Locking Condition:}
$$\mu > \frac{l}{\pi d_m}$$

or equivalently: $\alpha < \phi$ where $\alpha = \tan^{-1}(l/(\pi d_m))$ is lead angle and $\phi = \tan^{-1}(\mu)$ is friction angle

\subsection*{Threaded Fasteners (Bolts)}

\textbf{Tensile Stress Area:}
$$A_t = \frac{\pi}{4}\left(\frac{d - 0.9382p}{1}\right)^2$$

For Unified threads, approximately:
$$A_t \approx 0.7854\left(d - \frac{0.9743}{n}\right)^2$$

where $n$ is threads per inch

\textbf{Preload:}

Typical preload: $F_i = 0.75 F_{proof}$

$$F_{proof} = A_t S_p$$

where $S_p$ is proof strength

\textbf{Joint Stiffness:}

Bolt stiffness:
$$k_b = \frac{A_tE_b}{L_t}$$

where $L_t$ is grip length (threaded length under load)

Member stiffness (more complex):
$$k_m = \frac{E_m A_e}{L}$$

\textbf{External Load Distribution:}

Fraction to bolt:
$$C = \frac{k_b}{k_b + k_m}$$

Bolt force under external load $P$:
$$F_b = F_i + CP$$

Member force:
$$F_m = F_i(1 - C)P$$

\textbf{Fatigue Loading:}

Alternating stress in bolt:
$$\sigma_a = \frac{CP_a}{A_t}$$

Mean stress:
$$\sigma_m = \frac{F_i + CP_m}{A_t}$$

Apply fatigue criteria with these stresses

\subsection*{Bolt Torque}

\textbf{Tightening Torque:}
$$T = KFd$$

where:
- $K$ = nut factor (typically 0.15-0.20 for lubricated)
- $F$ = desired bolt tension
- $d$ = nominal diameter

\section*{Gears}

\subsection*{Gear Nomenclature}

\textbf{Basic Relationships:}
$$d = \frac{N}{P_d} = Nm$$

where:
- $d$ = pitch diameter
- $N$ = number of teeth
- $P_d$ = diametral pitch (teeth/inch)
- $m$ = module (mm/tooth)

\textbf{Velocity Ratio:}
$$VR = \frac{\omega_1}{\omega_2} = \frac{N_2}{N_1} = \frac{d_2}{d_1}$$

\textbf{Circular Pitch:}
$$p = \frac{\pi d}{N} = \frac{\pi}{P_d}$$

\textbf{Center Distance:}
$$C = \frac{d_1 + d_2}{2} = \frac{N_1 + N_2}{2P_d}$$

\subsection*{Spur Gears}

\textbf{Transmitted Force (tangential):}
$$W_t = \frac{T}{r} = \frac{P}{\omega r} = \frac{2P}{\omega d}$$

where $T$ is torque, $P$ is power, $\omega$ is angular velocity

\textbf{Radial Force:}
$$W_r = W_t\tan\phi$$

where $\phi$ is pressure angle (typically 20° or 25°)

\textbf{Lewis Equation (bending stress):}
$$\sigma = \frac{W_t}{FmY}$$

where:
- $F$ = face width
- $m$ = module
- $Y$ = Lewis form factor (depends on number of teeth)

\textbf{AGMA Bending Stress:}
$$\sigma = W_t K_o K_v K_s \frac{P_d}{F} \frac{K_m K_B}{J}$$

where $K$ factors account for:
- $K_o$ = overload
- $K_v$ = dynamic
- $K_s$ = size
- $K_m$ = load distribution
- $K_B$ = rim thickness
- $J$ = geometry factor

\textbf{Contact Stress (Hertzian):}
$$\sigma_c = C_p\sqrt{\frac{W_t K_o K_v K_s K_m}{F d_1} \frac{C_f}{I}}$$

where:
- $C_p$ = elastic coefficient
- $C_f$ = surface condition factor
- $I$ = geometry factor

\subsection*{Helical Gears}

Similar to spur gears but with additional axial force component:

\textbf{Axial Force:}
$$W_a = W_t\tan\psi$$

where $\psi$ is helix angle

\subsection*{Bevel Gears}

\textbf{Forces:}

Similar analysis but forces act at pitch cone angle

Decompose into tangential, radial, and axial components

\subsection*{Worm Gears}

\textbf{Velocity Ratio:}
$$VR = \frac{N_w}{N_g}$$

where $N_w$ is number of threads on worm, $N_g$ is teeth on gear

\textbf{Efficiency:}
$$e = \frac{\cos\phi_n - f\tan\lambda}{\cos\phi_n + f\cot\lambda}$$

where $\phi_n$ is normal pressure angle, $f$ is coefficient of friction, $\lambda$ is lead angle

Worm gears can be self-locking if $\lambda < \tan^{-1}(f)$

\section*{Bearings}

\subsection*{Rolling Contact Bearings}

\textbf{Basic Life Equation:}
$$L_{10} = \left(\frac{C}{P}\right)^a$$

where:
- $L_{10}$ = rating life (millions of revolutions for 90\% reliability)
- $C$ = basic dynamic load rating
- $P$ = equivalent dynamic load
- $a$ = 3 for ball bearings, 10/3 for roller bearings

\textbf{Life in Hours:}
$$L_{10h} = \frac{L_{10} \times 10^6}{60n}$$

where $n$ is rotational speed (rpm)

\textbf{Equivalent Dynamic Load:}

For radial bearings:
$$P = XF_r + YF_a$$

where:
- $F_r$ = radial load
- $F_a$ = axial (thrust) load
- $X$, $Y$ = radial and thrust factors (from manufacturer)

\textbf{Variable Loading:}

$$P_{eq} = \left[\frac{\sum(P_i^a n_i)}{\sum n_i}\right]^{1/a}$$

where $P_i$ are loads at different speeds $n_i$

\subsection*{Journal Bearings (Sliding Contact)}

\textbf{Petroff's Equation (light load):}
$$f = \frac{2\pi^2\mu N}{P}\left(\frac{r}{c}\right)$$

where:
- $f$ = coefficient of friction
- $\mu$ = absolute viscosity
- $N$ = shaft speed (rev/s)
- $P$ = bearing pressure
- $r$ = shaft radius
- $c$ = radial clearance

\textbf{Sommerfeld Number:}
$$S = \left(\frac{r}{c}\right)^2\frac{\mu N}{P}$$

Used to determine bearing performance from charts

\textbf{Minimum Film Thickness:}
$$h_0 = c(1 - \epsilon)$$

where $\epsilon$ is eccentricity ratio (from charts based on $S$)

\section*{Brakes and Clutches}

\subsection*{Friction Brakes}

\textbf{Torque Capacity:}
$$T = \mu F r$$

where:
- $\mu$ = coefficient of friction
- $F$ = normal force
- $r$ = effective radius

\textbf{Power Dissipation:}
$$P = T\omega$$

\textbf{Energy per Stop:}
$$E = \frac{1}{2}I\omega^2$$

where $I$ is moment of inertia of rotating mass

\subsection*{Band Brakes}

\textbf{Force Relationship:}
$$\frac{F_1}{F_2} = e^{\mu\beta}$$

where $\beta$ is wrap angle (radians)

\textbf{Torque:}
$$T = (F_1 - F_2)r$$

\section*{Keys and Pins}

\subsection*{Keys}

\textbf{Shear Stress:}
$$\tau = \frac{2T}{dLh}$$

\textbf{Bearing Stress:}
$$\sigma_b = \frac{4T}{dLh}$$

where:
- $T$ = torque
- $d$ = shaft diameter
- $L$ = key length
- $h$ = key height

\textbf{Standard Key Dimensions:}

Square key: $h = w = d/4$

\section*{Flywheels}

\textbf{Energy Storage:}
$$E = \frac{1}{2}I(\omega_{max}^2 - \omega_{min}^2)$$

\textbf{Coefficient of Fluctuation:}
$$C_s = \frac{\omega_{max} - \omega_{min}}{\omega_{avg}}$$

\textbf{Flywheel Design:}
$$I = \frac{E}{C_s\omega_{avg}^2}$$

\section*{Material Selection}

\textbf{Common Engineering Materials:}

\begin{center}
\begin{tabular}{lccc}
\toprule
\textbf{Material} & $S_y$ (MPa) & $S_{ut}$ (MPa) & \textbf{Applications} \\
\midrule
AISI 1020 (HR) & 210 & 380 & General purpose \\
AISI 1045 (CD) & 530 & 625 & Shafts, gears \\
AISI 4140 (Q\&T) & 655 & 855 & High strength \\
6061-T6 Al & 275 & 310 & Lightweight \\
Gray Cast Iron & - & 170 & Machine bases \\
\bottomrule
\end{tabular}
\end{center}

\section*{Quick Reference Formulas}

\textbf{Shaft Diameter (rough estimate):}
$$d \approx 2.5\sqrt[3]{\frac{P}{n}}$$

where $P$ is in kW, $n$ is in rpm, $d$ is in cm

\textbf{Gear Face Width:}
$$F = 3p \text{ to } 5p$$

where $p$ is circular pitch

\textbf{Spring Index Range:}
$$4 \leq C \leq 12$$

\end{document}