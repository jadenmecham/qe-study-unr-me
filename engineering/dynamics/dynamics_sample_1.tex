\documentclass[11pt]{article}
\usepackage[T1]{fontenc}
\usepackage{lmodern}
\usepackage[margin=1in]{geometry}
\usepackage{amsmath,amssymb}
\usepackage{array}
\usepackage{booktabs}

\pagestyle{empty}
\setlength{\parindent}{0pt}
\setlength{\parskip}{8pt}

\begin{document}

\begin{center}
\LARGE{\textbf{Dynamics Sample Problem 1}}
\end{center}

\vspace{10pt}

This problem is question 2.1 from the Fall 2023 QE exam:

\vspace{10pt}

A uniform heavy thin bar has one end attached to a frictionless pivot 
(that cannot apply any moment). The bar is dropped from a stationary 
horizontal position. When it reaches vertical
orientation, what is the velocity of its center of mass?

\section{Determine solution method}
Since this problem requires finding the velocity of an object, 
using work and energy methods will be most effective. 

\section{Identify knowns and unknowns}
\textbf{Knowns:}
\begin{itemize}
    \item The center of mass is located at the midpoint of the bar because
    it is uniform.
    \item The initial velocity of the bar is zero.
    \item The length of the bar is \( L \).
    \item The mass of the bar is \( m \).
    \item Gravitational acceleration is \( g \).
\end{itemize}
\textbf{Unknowns:}
\begin{itemize}
    \item The velocity of the center of mass when the bar reaches 
    vertical orientation, denoted as \( v \).
\end{itemize}

\section{Conservation of energy}
The total mechanical energy of the system is conserved since there are no 
non-conservative forces doing work (the pivot is frictionless).
\begin{gather}
    E_{initial} = E_{final}\\
    PE_{initial} + KE_{initial} = PE_{final} + KE_{final}
\end{gather}
At the initial position, there is no kinetic energy since the bar is 
stationary. At the final position, there is no potential energy since
the center of mass is at the lowest point. Then, the equation simplifies
to:
\begin{gather}
    PE_{initial} = KE_{final}
\end{gather}

\section{Expand and solve}

The initial potential energy of the bar when it is horizontal is given by:
\begin{gather}
    PE_{initial} = m g h
\end{gather}

\( h \) is the vertical distance the center of mass falls. Since the
center of mass is at the midpoint of the bar, \( h = \frac{L}{2} \).
Then:

\begin{gather}
    PE_{initial} = m g \left( \frac{L}{2} \right) = \frac{mgL}{2} 
\end{gather}

The final kinetic energy of the bar when it is vertical is given by:
\begin{gather}
    KE_{final} = \frac{1}{2} I \omega^2
\end{gather}

Where \( I \) is the mass moment of inertia of the bar. This is given by:
\begin{gather}
    I = \frac{1}{3} m L^2
\end{gather}

Do not confuse this with the moment of inertia about the center of mass. 
Here, we use mass moment of inertia of the bar rotating about a pivot
point at the end of the bar, not about its center of mass. Substitute in 
\( I \) into the kinetic energy equation. 
\begin{gather}
    KE_{final} = \frac{1}{2} \left( \frac{1}{3} m L^2 \right) \omega^2 = \frac{1}{6} m L^2 \omega^2
\end{gather}

Now, substitute the expressions for initial potential energy and final
kinetic energy into the conservation of energy equation.
\begin{gather}
    \frac{mgL}{2} = \frac{1}{6} m L^2 \omega^2
\end{gather}

Next, solve for \( \omega \).
\begin{gather}
    3mgL = mL^2\omega^2\\
    3gL = L^2\omega^2\\
    \omega^2 = \frac{3g}{L}\\
    \omega = \sqrt{\frac{3g}{L}}
\end{gather}

Finally, find the linear velocity of the center of mass using the 
relationship between angular velocity and linear velocity.
\begin{gather}
    v = \omega r
\end{gather}

Where \( r \) is the distance from the pivot to the center of mass, 
which is \( \frac{L}{2} \). Then:

\begin{gather}
    v = \frac{L}{2} \sqrt{\frac{3g}{L}}\\
    \boxed{v = \sqrt{\frac{3gL}{4}}}
\end{gather}

Nice.

\end{document}