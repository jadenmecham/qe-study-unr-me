\documentclass[11pt]{article}
\usepackage[T1]{fontenc}
\usepackage{lmodern}
\usepackage[margin=1in]{geometry}
\usepackage{amsmath,amssymb}
\usepackage{array}
\usepackage{booktabs}

\pagestyle{empty}
\setlength{\parindent}{0pt}
\setlength{\parskip}{8pt}

\begin{document}

\begin{center}
\LARGE{\textbf{Dynamics and Vibrations - QE Equation Sheet}}
\end{center}

\vspace{10pt}

\section*{Kinematics of Particles}

\subsection*{Rectilinear Motion}

\textbf{Position, Velocity, Acceleration:}
$$v = \frac{dx}{dt} = \dot{x}, \quad a = \frac{dv}{dt} = \ddot{x}$$

\textbf{Alternative forms:}
$$a = v\frac{dv}{dx}, \quad v\,dv = a\,dx$$

\textbf{Constant Acceleration:}
$$v = v_0 + at$$
$$x = x_0 + v_0t + \frac{1}{2}at^2$$
$$v^2 = v_0^2 + 2a(x - x_0)$$

\subsection*{Curvilinear Motion}

\textbf{Cartesian Coordinates:}
$$\mathbf{r} = x\mathbf{i} + y\mathbf{j} + z\mathbf{k}$$
$$\mathbf{v} = \dot{x}\mathbf{i} + \dot{y}\mathbf{j} + \dot{z}\mathbf{k}$$
$$\mathbf{a} = \ddot{x}\mathbf{i} + \ddot{y}\mathbf{j} + \ddot{z}\mathbf{k}$$

\textbf{Path Coordinates (Normal-Tangential):}
$$\mathbf{v} = v\hat{\mathbf{e}}_t$$
$$\mathbf{a} = \dot{v}\hat{\mathbf{e}}_t + \frac{v^2}{\rho}\hat{\mathbf{e}}_n$$

where $\rho$ is the radius of curvature

\textbf{Polar Coordinates:}
$$\mathbf{r} = r\hat{\mathbf{e}}_r$$
$$\mathbf{v} = \dot{r}\hat{\mathbf{e}}_r + r\dot{\theta}\hat{\mathbf{e}}_\theta$$
$$\mathbf{a} = (\ddot{r} - r\dot{\theta}^2)\hat{\mathbf{e}}_r + (r\ddot{\theta} + 2\dot{r}\dot{\theta})\hat{\mathbf{e}}_\theta$$

\textbf{Cylindrical Coordinates:}
$$\mathbf{v} = \dot{r}\hat{\mathbf{e}}_r + r\dot{\theta}\hat{\mathbf{e}}_\theta + \dot{z}\hat{\mathbf{e}}_z$$
$$\mathbf{a} = (\ddot{r} - r\dot{\theta}^2)\hat{\mathbf{e}}_r + (r\ddot{\theta} + 2\dot{r}\dot{\theta})\hat{\mathbf{e}}_\theta + \ddot{z}\hat{\mathbf{e}}_z$$

\subsection*{Relative Motion}

\textbf{Translating Reference Frame:}
$$\mathbf{v}_B = \mathbf{v}_A + \mathbf{v}_{B/A}$$
$$\mathbf{a}_B = \mathbf{a}_A + \mathbf{a}_{B/A}$$

\textbf{Rotating Reference Frame:}
$$\mathbf{v}_B = \mathbf{v}_A + \boldsymbol{\omega} \times \mathbf{r}_{B/A} + (\mathbf{v}_{B/A})_{rel}$$
$$\mathbf{a}_B = \mathbf{a}_A + \boldsymbol{\alpha} \times \mathbf{r}_{B/A} + \boldsymbol{\omega} \times (\boldsymbol{\omega} \times \mathbf{r}_{B/A}) + 2\boldsymbol{\omega} \times (\mathbf{v}_{B/A})_{rel} + (\mathbf{a}_{B/A})_{rel}$$

where $2\boldsymbol{\omega} \times (\mathbf{v}_{B/A})_{rel}$ is the Coriolis acceleration

\section*{Rigid Body Kinematics}

\subsection*{General Planar Motion}

\textbf{Velocity:}
$$\mathbf{v}_B = \mathbf{v}_A + \boldsymbol{\omega} \times \mathbf{r}_{B/A}$$

In 2D:
$$v_B = v_A + \omega r_{B/A} \perp \mathbf{r}_{B/A}$$

\textbf{Acceleration:}
$$\mathbf{a}_B = \mathbf{a}_A + \boldsymbol{\alpha} \times \mathbf{r}_{B/A} + \boldsymbol{\omega} \times (\boldsymbol{\omega} \times \mathbf{r}_{B/A})$$

In 2D:
$$\mathbf{a}_B = \mathbf{a}_A + \alpha r_{B/A}(\perp \mathbf{r}_{B/A}) - \omega^2 r_{B/A}(\parallel \mathbf{r}_{B/A})$$

\subsection*{Rolling Without Slipping}

For a wheel of radius $R$ rolling on a fixed surface:
$$v_{\text{center}} = \omega R$$
$$a_{\text{center}} = \alpha R$$

Point of contact has zero velocity (instantaneous center)

\subsection*{Instantaneous Center of Zero Velocity (IC)}

For planar motion, there exists a point IC such that $\mathbf{v}_{\text{IC}} = \mathbf{0}$ at that instant

\textbf{Finding IC:}
\begin{itemize}
\item Draw velocity vectors of two points
\item Draw perpendiculars to each velocity
\item IC is at the intersection
\end{itemize}

\textbf{Using IC:}
$$v_A = \omega \cdot r_{A/IC}$$

where $r_{A/IC}$ is distance from IC to point A

\textbf{Note:} IC generally has non-zero acceleration

\subsection*{Rotation About a Fixed Axis}

$$\boldsymbol{\omega} = \dot{\theta}\hat{\mathbf{k}}$$
$$\boldsymbol{\alpha} = \ddot{\theta}\hat{\mathbf{k}}$$

For a point at distance $r$ from the axis:
$$v = r\omega, \quad a_t = r\alpha, \quad a_n = r\omega^2$$

\section*{Mass Properties}

\subsection*{Center of Mass}

\textbf{Discrete particles:}
$$\mathbf{r}_G = \frac{\sum m_i \mathbf{r}_i}{\sum m_i} = \frac{\sum m_i \mathbf{r}_i}{M}$$

\textbf{Continuous body:}
$$\mathbf{r}_G = \frac{1}{M}\int \mathbf{r}\,dm$$

In coordinates:
$$x_G = \frac{1}{M}\int x\,dm, \quad y_G = \frac{1}{M}\int y\,dm, \quad z_G = \frac{1}{M}\int z\,dm$$

\subsection*{Mass Moment of Inertia}

\textbf{Definition:}
$$I = \int r^2\,dm$$

where $r$ is perpendicular distance from axis of rotation

\textbf{Parallel Axis Theorem:}
$$I_O = I_G + Md^2$$

where $d$ is the distance between parallel axes through $O$ and $G$

\textbf{Radius of Gyration:}
$$I = Mk^2 \quad \Rightarrow \quad k = \sqrt{\frac{I}{M}}$$

\subsection*{Common Moments of Inertia}

\textbf{Slender Rod (length $L$, mass $m$):}
$$I_{\text{center}} = \frac{1}{12}mL^2, \quad I_{\text{end}} = \frac{1}{3}mL^2$$

\textbf{Thin Disk/Cylinder (radius $R$, mass $m$):}
$$I_{\text{center, perpendicular}} = \frac{1}{2}mR^2$$
$$I_{\text{center, diameter}} = \frac{1}{4}mR^2$$

\textbf{Thin Ring (radius $R$, mass $m$):}
$$I_{\text{center}} = mR^2$$

\textbf{Sphere (radius $R$, mass $m$):}
$$I_{\text{center}} = \frac{2}{5}mR^2$$

\textbf{Rectangular Plate (sides $a \times b$, mass $m$):}
$$I_{\text{center, perpendicular}} = \frac{1}{12}m(a^2 + b^2)$$

\textbf{Thin Plate (perpendicular axis theorem):}
$$I_z = I_x + I_y$$

\section*{Kinetics: Newton-Euler Equations}

\subsection*{Particle Kinetics}

\textbf{Newton's Second Law:}
$$\sum \mathbf{F} = m\mathbf{a}$$

In component form:
$$\sum F_x = ma_x, \quad \sum F_y = ma_y, \quad \sum F_z = ma_z$$

\subsection*{Rigid Body Kinetics (Planar Motion)}

\textbf{Equation of Motion:}
$$\sum \mathbf{F} = m\mathbf{a}_G$$
$$\sum M_G = I_G\alpha$$

\textbf{About a Fixed Point O:}
$$\sum M_O = I_O\alpha$$

\textbf{About Point P (not fixed, not at G):}
$$\sum M_P = I_G\alpha + \mathbf{r}_{G/P} \times m\mathbf{a}_G$$

\textbf{Component Form (2D):}
$$\sum F_x = m(a_G)_x$$
$$\sum F_y = m(a_G)_y$$
$$\sum M_G = I_G\alpha$$

\section*{Work and Energy}

\subsection*{Work}

\textbf{Definition:}
$$W = \int \mathbf{F} \cdot d\mathbf{r}$$

\textbf{Constant Force:}
$$W = F \cdot d \cos\theta$$

\textbf{Spring Force:}
$$W = -\frac{1}{2}k(x_2^2 - x_1^2)$$

\textbf{Weight:}
$$W = -mg(y_2 - y_1) = -mg\Delta h$$

\textbf{Moment/Torque:}
$$W = \int M\,d\theta$$

For constant moment:
$$W = M(\theta_2 - \theta_1)$$

\subsection*{Kinetic Energy}

\textbf{Particle:}
$$T = \frac{1}{2}mv^2$$

\textbf{Rigid Body (General Planar Motion):}
$$T = \frac{1}{2}mv_G^2 + \frac{1}{2}I_G\omega^2$$

\textbf{Rigid Body (Rotation about Fixed Axis):}
$$T = \frac{1}{2}I_O\omega^2$$

\textbf{Rolling Body:}
$$T = \frac{1}{2}mv_G^2 + \frac{1}{2}I_G\omega^2 = \frac{1}{2}(m + \frac{I_G}{R^2})v_G^2$$

\subsection*{Potential Energy}

\textbf{Gravitational:}
$$V_g = mgh$$

where $h$ is height above reference datum

\textbf{Elastic (Spring):}
$$V_e = \frac{1}{2}kx^2$$

where $x$ is deformation from unstretched position

\subsection*{Work-Energy Theorem}

$$T_1 + \sum U_{1-2} = T_2$$

or

$$T_1 + V_1 + W_{\text{nc}} = T_2 + V_2$$

where $W_{\text{nc}}$ is work by non-conservative forces

\textbf{Conservation of Energy:}

If only conservative forces do work:
$$T_1 + V_1 = T_2 + V_2$$
$$E = T + V = \text{constant}$$

\subsection*{Power}

\textbf{Definition:}
$$P = \frac{dW}{dt} = \mathbf{F} \cdot \mathbf{v}$$

\textbf{For rotation:}
$$P = M\omega$$

\textbf{Efficiency:}
$$\eta = \frac{P_{\text{out}}}{P_{\text{in}}}$$

\section*{Impulse and Momentum}

\subsection*{Linear Impulse-Momentum}

\textbf{Linear Momentum:}
$$\mathbf{L} = m\mathbf{v}$$

\textbf{Impulse-Momentum Principle:}
$$\int_{t_1}^{t_2} \sum \mathbf{F}\,dt = m\mathbf{v}_2 - m\mathbf{v}_1$$

\textbf{Conservation of Linear Momentum:}

If $\sum \mathbf{F}_{\text{ext}} = 0$:
$$m_1\mathbf{v}_1 + m_2\mathbf{v}_2 = \text{constant}$$

\subsection*{Angular Impulse-Momentum}

\textbf{Angular Momentum about Point O:}

For a particle:
$$\mathbf{H}_O = \mathbf{r} \times m\mathbf{v}$$

For a rigid body:
$$\mathbf{H}_O = I_O\boldsymbol{\omega}$$

For a rigid body about G:
$$\mathbf{H}_G = I_G\boldsymbol{\omega}$$

For a rigid body about arbitrary point O:
$$\mathbf{H}_O = I_G\boldsymbol{\omega} + \mathbf{r}_{G/O} \times m\mathbf{v}_G$$

\textbf{Angular Impulse-Momentum Principle:}
$$\int_{t_1}^{t_2} \sum M_O\,dt = (H_O)_2 - (H_O)_1$$

\textbf{Conservation of Angular Momentum:}

If $\sum M_O = 0$:
$$H_O = \text{constant}$$

\subsection*{Impact and Collisions}

\textbf{Coefficient of Restitution:}
$$e = \frac{(v_B)_2 - (v_A)_2}{(v_A)_1 - (v_B)_1}$$

where subscripts $A$ and $B$ denote two bodies, and subscripts 1 and 2 denote before and after impact

\textbf{Types of Collisions:}
\begin{itemize}
\item $e = 1$: Perfectly elastic (kinetic energy conserved)
\item $e = 0$: Perfectly plastic (bodies stick together)
\item $0 < e < 1$: Real collisions (some energy lost)
\end{itemize}

\textbf{For direct central impact:}

Conservation of momentum:
$$m_A(v_A)_1 + m_B(v_B)_1 = m_A(v_A)_2 + m_B(v_B)_2$$

Restitution:
$$(v_B)_2 - (v_A)_2 = -e[(v_B)_1 - (v_A)_1]$$

\section*{Special Topics}

\subsection*{Systems of Particles}

\textbf{Center of Mass Motion:}
$$\sum \mathbf{F}_{\text{ext}} = M\mathbf{a}_G$$

\textbf{Total Linear Momentum:}
$$\mathbf{L}_{\text{total}} = M\mathbf{v}_G$$

\textbf{Angular Momentum about G:}
$$\mathbf{H}_G = \sum (\mathbf{r}_i - \mathbf{r}_G) \times m_i\mathbf{v}_i$$

\subsection*{Variable Mass Systems}

\textbf{Thrust Equation (Rocket):}
$$\sum F_{\text{ext}} = m\frac{dv}{dt} - v_{\text{rel}}\frac{dm}{dt}$$

where $v_{\text{rel}}$ is exhaust velocity relative to rocket

\subsection*{Constraints and Dependent Motion}

\textbf{Cable Constraint:}

For a system with cables/ropes, write constraint equation based on constant total length:
$$L_{\text{total}} = \sum L_i = \text{constant}$$

Differentiate to relate velocities:
$$\sum \frac{dL_i}{dt} = 0$$

Differentiate again to relate accelerations:
$$\sum \frac{d^2L_i}{dt^2} = 0$$

\textbf{Rolling Constraint:}

For pure rolling (no slip):
$$v = R\omega, \quad a = R\alpha$$

\textbf{Gears:}

For meshing gears:
$$r_1\omega_1 = r_2\omega_2$$
$$r_1\alpha_1 = r_2\alpha_2$$

\section*{Problem-Solving Strategies}

\subsection*{Choosing the Right Approach}

\textbf{Use Newton-Euler when:}
\begin{itemize}
\item Need to find forces/reactions
\item Acceleration is needed
\item Problem involves time explicitly
\end{itemize}

\textbf{Use Work-Energy when:}
\begin{itemize}
\item Only interested in velocities (not time)
\item Forces do work over a distance
\item Springs are involved
\end{itemize}

\textbf{Use Impulse-Momentum when:}
\begin{itemize}
\item Forces act over time intervals
\item Collisions/impacts occur
\item Conservation principles apply
\end{itemize}

\subsection*{Common Steps}

\textbf{For Newton-Euler Problems:}
\begin{enumerate}
\item Draw free body diagram (FBD)
\item Choose coordinate system
\item Write kinematic relationships
\item Apply $\sum F = ma$ and $\sum M = I\alpha$
\item Solve system of equations
\end{enumerate}

\textbf{For Energy Problems:}
\begin{enumerate}
\item Identify datum for potential energy
\item Write $T_1 + V_1$ at initial state
\item Write $T_2 + V_2$ at final state
\item Calculate work by non-conservative forces
\item Apply $T_1 + V_1 + W_{nc} = T_2 + V_2$
\end{enumerate}

\end{document}