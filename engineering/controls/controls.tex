\documentclass[11pt]{article}
\usepackage[T1]{fontenc}
\usepackage{lmodern}
\usepackage[margin=1in]{geometry}
\usepackage{amsmath,amssymb}
\usepackage{array}
\usepackage{booktabs}

\pagestyle{empty}
\setlength{\parindent}{0pt}
\setlength{\parskip}{8pt}

\begin{document}

\begin{center}
\LARGE{\textbf{Systems and Controls}}
\end{center}

\vspace{10pt}

\section*{Laplace Transform for Control Systems}

\textbf{Definition:}
$$\mathcal{L}\{f(t)\} = F(s) = \int_0^\infty f(t)e^{-st}\,dt$$

\textbf{Common Transforms:}
$$\mathcal{L}\{1\} = \frac{1}{s}, \quad \mathcal{L}\{t\} = \frac{1}{s^2}, \quad \mathcal{L}\{e^{at}\} = \frac{1}{s-a}$$

$$\mathcal{L}\{\sin(\omega t)\} = \frac{\omega}{s^2 + \omega^2}, \quad \mathcal{L}\{\cos(\omega t)\} = \frac{s}{s^2 + \omega^2}$$

\textbf{Derivatives:}
$$\mathcal{L}\{\dot{f}(t)\} = sF(s) - f(0)$$
$$\mathcal{L}\{\ddot{f}(t)\} = s^2F(s) - sf(0) - \dot{f}(0)$$

\textbf{Final Value Theorem:}
$$\lim_{t\to\infty} f(t) = \lim_{s\to 0} sF(s)$$
(Valid only if $f(t)$ has a final value, i.e., system is stable)

\textbf{Initial Value Theorem:}
$$\lim_{t\to 0^+} f(t) = \lim_{s\to\infty} sF(s)$$

\section*{Transfer Functions}

\textbf{Definition:}
$$G(s) = \frac{Y(s)}{U(s)} = \frac{\text{Output}}{\text{Input}}$$

assuming zero initial conditions

\subsection*{Standard Transfer Function Forms}

\textbf{First-Order System:}
$$G(s) = \frac{K}{\tau s + 1}$$

where $K$ is DC gain and $\tau$ is time constant

\textbf{Second-Order System:}
$$G(s) = \frac{K\omega_n^2}{s^2 + 2\zeta\omega_n s + \omega_n^2}$$

where $\omega_n$ is natural frequency and $\zeta$ is damping ratio

Alternative form:
$$G(s) = \frac{K}{s^2 + 2\zeta\omega_n s + \omega_n^2}$$

\subsection*{System Modeling}

\textbf{Mechanical System (Mass-Spring-Damper):}
$$m\ddot{x} + c\dot{x} + kx = f(t)$$

Transfer function:
$$G(s) = \frac{X(s)}{F(s)} = \frac{1}{ms^2 + cs + k}$$

\textbf{RLC Circuit:}
$$L\ddot{q} + R\dot{q} + \frac{1}{C}q = v(t)$$

Transfer function:
$$G(s) = \frac{Q(s)}{V(s)} = \frac{1}{Ls^2 + Rs + 1/C}$$

\subsection*{Mechanical-Electrical Analogies}

\begin{center}
\begin{tabular}{cc}
\toprule
\textbf{Mechanical} & \textbf{Electrical} \\
\midrule
Force $F$ & Voltage $V$ \\
Velocity $v$ & Current $I$ \\
Mass $m$ & Inductance $L$ \\
Damper $c$ & Resistance $R$ \\
Spring $k$ & $1/C$ (Capacitance) \\
\bottomrule
\end{tabular}
\end{center}

\section*{Block Diagram Algebra}

\textbf{Series (Cascade):}
$$G_{\text{eq}}(s) = G_1(s)G_2(s)$$

\textbf{Parallel:}
$$G_{\text{eq}}(s) = G_1(s) + G_2(s)$$

\textbf{Feedback (Negative):}
$$G_{cl}(s) = \frac{G(s)}{1 + G(s)H(s)}$$

where $G(s)$ is forward path and $H(s)$ is feedback path

\textbf{Feedback (Positive):}
$$G_{cl}(s) = \frac{G(s)}{1 - G(s)H(s)}$$

\textbf{Unity Feedback:} $H(s) = 1$
$$G_{cl}(s) = \frac{G(s)}{1 + G(s)}$$

\subsection*{Moving Summing Junctions and Pickoff Points}

\textbf{Moving summing junction past a block $G$:}
- Move forward: Divide by $G$
- Move backward: Multiply by $G$

\textbf{Moving pickoff point past a block $G$:}
- Move forward: Multiply by $G$
- Move backward: Divide by $G$

\section*{Time Response Analysis}

\subsection*{First-Order System Response}

For $G(s) = \frac{K}{\tau s + 1}$ with unit step input:

\textbf{Time Response:}
$$y(t) = K(1 - e^{-t/\tau})$$

\textbf{Time Constant:} $\tau$
- Time to reach 63.2\% of final value
- $y(\tau) = 0.632K$

\textbf{Settling Time (2\% criterion):}
$$T_s = 4\tau$$

\subsection*{Second-Order System Response}

For $G(s) = \frac{\omega_n^2}{s^2 + 2\zeta\omega_n s + \omega_n^2}$ with unit step input:

\textbf{Underdamped} ($0 < \zeta < 1$):
$$y(t) = 1 - \frac{e^{-\zeta\omega_n t}}{\sqrt{1-\zeta^2}}\sin(\omega_d t + \phi)$$

where $\omega_d = \omega_n\sqrt{1-\zeta^2}$ and $\phi = \cos^{-1}(\zeta)$

\textbf{Performance Specifications:}

\textit{Rise Time (0\% to 100\%):}
$$T_r \approx \frac{1.8}{\omega_n}$$

\textit{Peak Time:}
$$T_p = \frac{\pi}{\omega_d} = \frac{\pi}{\omega_n\sqrt{1-\zeta^2}}$$

\textit{Percent Overshoot:}
$$M_p = e^{-\frac{\zeta\pi}{\sqrt{1-\zeta^2}}} \times 100\%$$

\textit{Settling Time (2\% criterion):}
$$T_s \approx \frac{4}{\zeta\omega_n}$$

\textit{Settling Time (5\% criterion):}
$$T_s \approx \frac{3}{\zeta\omega_n}$$

\textbf{Damping Ratio from Overshoot:}
$$\zeta = \frac{-\ln(M_p/100)}{\sqrt{\pi^2 + \ln^2(M_p/100)}}$$

For small overshoot:
$$\zeta \approx 1 - \frac{M_p}{100}$$

\subsection*{Dominant Pole Approximation}

For higher-order systems, if one pole (or pair) is much closer to imaginary axis than others, system behaves approximately like first or second-order system with those dominant poles.

\section*{Steady-State Error}

\subsection*{Error Definition}

$$E(s) = R(s) - Y(s) = \frac{R(s)}{1 + G(s)H(s)}$$

\textbf{Steady-State Error:}
$$e_{ss} = \lim_{t\to\infty} e(t) = \lim_{s\to 0} sE(s)$$

\subsection*{Error Constants (Unity Feedback)}

\textbf{Position Error Constant:}
$$K_p = \lim_{s\to 0} G(s)$$

\textbf{Velocity Error Constant:}
$$K_v = \lim_{s\to 0} sG(s)$$

\textbf{Acceleration Error Constant:}
$$K_a = \lim_{s\to 0} s^2G(s)$$

\subsection*{Steady-State Errors for Standard Inputs}

\begin{center}
\begin{tabular}{cccc}
\toprule
\textbf{Input} & \textbf{Type 0} & \textbf{Type 1} & \textbf{Type 2} \\
\midrule
Step: $R(s) = \frac{1}{s}$ & $\frac{1}{1+K_p}$ & $0$ & $0$ \\[6pt]
Ramp: $R(s) = \frac{1}{s^2}$ & $\infty$ & $\frac{1}{K_v}$ & $0$ \\[6pt]
Parabolic: $R(s) = \frac{1}{s^3}$ & $\infty$ & $\infty$ & $\frac{1}{K_a}$ \\
\bottomrule
\end{tabular}
\end{center}

\textbf{System Type:}

Number of integrators (poles at $s = 0$) in open-loop transfer function $G(s)H(s)$

For $G(s)H(s) = \frac{K(s+z_1)\cdots}{s^N(s+p_1)\cdots}$, Type = $N$

\section*{Stability Analysis}

\subsection*{Characteristic Equation}

For closed-loop system:
$$1 + G(s)H(s) = 0$$

or equivalently, denominator of $\frac{G(s)}{1 + G(s)H(s)} = 0$

\textbf{Stability Requirement:}

All roots of characteristic equation must have negative real parts (left half-plane)

\subsection*{Routh-Hurwitz Stability Criterion}

For characteristic equation:
$$a_ns^n + a_{n-1}s^{n-1} + \cdots + a_1s + a_0 = 0$$

\textbf{Routh Array:}
\begin{center}
\begin{tabular}{cccc}
$s^n$ & $a_n$ & $a_{n-2}$ & $a_{n-4}$ \\
$s^{n-1}$ & $a_{n-1}$ & $a_{n-3}$ & $a_{n-5}$ \\
$s^{n-2}$ & $b_1$ & $b_2$ & $b_3$ \\
$s^{n-3}$ & $c_1$ & $c_2$ & $c_3$ \\
$\vdots$ & $\vdots$ & $\vdots$ & $\vdots$ \\
$s^0$ & $h_1$ & & \\
\end{tabular}
\end{center}

where:
$$b_1 = \frac{a_{n-1}a_{n-2} - a_na_{n-3}}{a_{n-1}}, \quad b_2 = \frac{a_{n-1}a_{n-4} - a_na_{n-5}}{a_{n-1}}$$

$$c_1 = \frac{b_1a_{n-3} - a_{n-1}b_2}{b_1}, \quad c_2 = \frac{b_1a_{n-5} - a_{n-1}b_3}{b_1}$$

\textbf{Stability Criterion:}

System is stable if and only if all elements in the first column have the same sign (no sign changes)

\textbf{Number of RHP poles:} = Number of sign changes in first column

\textbf{Special Cases:}
\begin{itemize}
\item If first element in row is zero but others aren't: Replace with small $\epsilon > 0$ and continue
\item If entire row is zero: Indicates roots on imaginary axis (marginal stability)
\end{itemize}

\section*{Root Locus Method}

\textbf{Root Locus:} Plot of closed-loop poles as gain $K$ varies from $0$ to $\infty$

For system: $1 + KG(s)H(s) = 0$

\subsection*{Root Locus Construction Rules}

\textbf{Rule 1 - Number of Branches:}

Number of branches = Number of poles of $G(s)H(s) = n$

\textbf{Rule 2 - Starting and Ending Points:}

Loci start ($K = 0$) at poles of $G(s)H(s)$

Loci end ($K = \infty$) at zeros of $G(s)H(s)$

\textbf{Rule 3 - Real Axis Segments:}

Locus exists on real axis to the left of an odd number of real poles and zeros

\textbf{Rule 4 - Asymptotes:}

Number of asymptotes: $n - m$ (poles minus zeros)

Asymptote angles:
$$\theta_k = \frac{(2k + 1)\pi}{n - m}, \quad k = 0, 1, 2, \ldots, (n-m-1)$$

Centroid (intersection point):
$$\sigma_a = \frac{\sum \text{poles} - \sum \text{zeros}}{n - m}$$

\textbf{Rule 5 - Breakaway/Break-in Points:}

Solve: $\frac{dK}{ds} = 0$ or $\frac{d}{ds}[G(s)H(s)] = 0$

Points where locus leaves (breakaway) or enters (break-in) real axis

\textbf{Rule 6 - Imaginary Axis Crossings:}

Use Routh-Hurwitz to find:
- Value of $K$ at crossing
- Frequency $\omega$ at crossing

\textbf{Rule 7 - Angle of Departure/Arrival:}

From complex pole:
$$\theta_d = 180° - \sum \text{angles from zeros} + \sum \text{angles from other poles}$$

To complex zero:
$$\theta_a = 180° + \sum \text{angles from zeros} - \sum \text{angles from poles}$$

\subsection*{Angle and Magnitude Conditions}

\textbf{Angle Condition:}
$$\angle G(s)H(s) = (2k + 1) \cdot 180°, \quad k = 0, \pm 1, \pm 2, \ldots$$

\textbf{Magnitude Condition:}
$$K = \frac{1}{|G(s)H(s)|}$$

\section*{Frequency Response Analysis}

\subsection*{Frequency Response}

Substitute $s = j\omega$ into transfer function:
$$G(j\omega) = |G(j\omega)|e^{j\angle G(j\omega)}$$

\textbf{Magnitude:}
$$|G(j\omega)| = \sqrt{\text{Re}^2 + \text{Im}^2}$$

\textbf{Phase:}
$$\angle G(j\omega) = \tan^{-1}\left(\frac{\text{Im}}{\text{Re}}\right)$$

\subsection*{Bode Plots}

\textbf{Magnitude Plot:} $20\log_{10}|G(j\omega)|$ (dB) vs $\log(\omega)$

\textbf{Phase Plot:} $\angle G(j\omega)$ (degrees) vs $\log(\omega)$

\subsection*{Bode Plot Construction}

\textbf{Constant $K$:}
- Magnitude: $20\log_{10}K$ dB (horizontal line)
- Phase: $0°$ if $K > 0$, $-180°$ if $K < 0$

\textbf{Pole at origin $\frac{1}{s}$:}
- Magnitude: $-20$ dB/decade slope
- Phase: $-90°$

\textbf{Zero at origin $s$:}
- Magnitude: $+20$ dB/decade slope
- Phase: $+90°$

\textbf{First-order pole $\frac{1}{\tau s + 1}$:}

Break frequency: $\omega_b = 1/\tau$

- Magnitude: $0$ dB for $\omega \ll \omega_b$, $-20$ dB/decade for $\omega \gg \omega_b$
- Phase: $0°$ for $\omega \ll \omega_b$, $-45°$ at $\omega_b$, $-90°$ for $\omega \gg \omega_b$

\textbf{First-order zero $(\tau s + 1)$:}

Break frequency: $\omega_b = 1/\tau$

- Magnitude: $0$ dB for $\omega \ll \omega_b$, $+20$ dB/decade for $\omega \gg \omega_b$
- Phase: $0°$ for $\omega \ll \omega_b$, $+45°$ at $\omega_b$, $+90°$ for $\omega \gg \omega_b$

\textbf{Second-order pole $\frac{1}{s^2/\omega_n^2 + 2\zeta s/\omega_n + 1}$:}

Break frequency: $\omega_b = \omega_n$

- Magnitude: $0$ dB for $\omega \ll \omega_n$, $-40$ dB/decade for $\omega \gg \omega_n$
- Resonant peak for small $\zeta$: $M_r \approx \frac{1}{2\zeta}$ at $\omega \approx \omega_n$
- Phase: $0°$ for $\omega \ll \omega_n$, $-90°$ at $\omega_n$, $-180°$ for $\omega \gg \omega_n$

\subsection*{Stability Margins}

\textbf{Gain Margin (GM):}

At phase crossover frequency $\omega_{pc}$ (where $\angle G(j\omega) = -180°$):
$$\text{GM (dB)} = -20\log_{10}|G(j\omega_{pc})|$$

Stable if GM $> 0$ dB

\textbf{Phase Margin (PM):}

At gain crossover frequency $\omega_{gc}$ (where $|G(j\omega)| = 1$ or $0$ dB):
$$\text{PM} = 180° + \angle G(j\omega_{gc})$$

Stable if PM $> 0°$

\textbf{Typical Design Criteria:}
- GM $\geq 6$ dB
- PM $\geq 30°$ to $60°$ (typically $45°$ for good performance)

\subsection*{Nyquist Stability Criterion}

\textbf{Nyquist Plot:} Polar plot of $G(j\omega)H(j\omega)$ for $\omega: 0 \to \infty$

\textbf{Nyquist Stability Criterion:}

System is stable if:
$$Z = N + P = 0$$

where:
- $Z$ = number of closed-loop RHP poles
- $P$ = number of open-loop RHP poles
- $N$ = number of clockwise encirclements of $-1 + j0$ point

For stable open-loop system ($P = 0$): System is stable if Nyquist plot does not encircle $-1 + j0$ point

\textbf{Simplified for Stable Open-Loop:}

Stable if Nyquist plot passes to the left of $-1 + j0$ point

\section*{Controllers and Compensation}

\subsection*{Proportional (P) Controller}

$$G_c(s) = K_p$$

\textbf{Effects:}
- Increases gain
- Reduces steady-state error
- Can destabilize system if too high
- Does NOT eliminate steady-state error for step input (Type 0)

\subsection*{Proportional-Integral (PI) Controller}

$$G_c(s) = K_p\left(1 + \frac{1}{T_i s}\right) = K_p\frac{T_i s + 1}{T_i s}$$

\textbf{Effects:}
- Eliminates steady-state error for step input
- Increases system type by 1
- May slow down response
- Can reduce stability margins

\subsection*{Proportional-Derivative (PD) Controller}

$$G_c(s) = K_p(1 + T_d s)$$

\textbf{Effects:}
- Improves transient response
- Increases damping
- Improves stability margins
- Does NOT affect steady-state error
- Amplifies high-frequency noise

\subsection*{Proportional-Integral-Derivative (PID) Controller}

$$G_c(s) = K_p\left(1 + \frac{1}{T_i s} + T_d s\right)$$

Practical form (with derivative filter):
$$G_c(s) = K_p + \frac{K_i}{s} + \frac{K_d s}{1 + \tau_f s}$$

\textbf{Effects:}
- Combines benefits of P, I, and D
- Zero steady-state error
- Good transient response
- Improved stability

\textbf{Tuning Methods:}
- Ziegler-Nichols
- Cohen-Coon
- Trial and error
- Software optimization

\subsection*{Lead Compensator}

$$G_c(s) = K_c\frac{s + z}{s + p}, \quad z < p$$

\textbf{Effects:}
- Increases phase margin
- Improves transient response
- Increases bandwidth
- Used for phase lead

\textbf{Maximum Phase Lead:}
$$\phi_{max} = \sin^{-1}\left(\frac{p/z - 1}{p/z + 1}\right)$$

Occurs at $\omega_m = \sqrt{zp}$

\subsection*{Lag Compensator}

$$G_c(s) = K_c\frac{s + z}{s + p}, \quad z > p$$

\textbf{Effects:}
- Increases DC gain
- Reduces steady-state error
- Decreases bandwidth
- May slow response

\subsection*{Lead-Lag Compensator}

$$G_c(s) = K_c\frac{(s + z_1)(s + z_2)}{(s + p_1)(s + p_2)}$$

Combines benefits of both lead and lag compensation

\section*{State-Space Representation}

\subsection*{State Equations}

$$\dot{\mathbf{x}}(t) = \mathbf{A}\mathbf{x}(t) + \mathbf{B}u(t)$$
$$y(t) = \mathbf{C}\mathbf{x}(t) + \mathbf{D}u(t)$$

where:
- $\mathbf{x}$ is state vector ($n \times 1$)
- $u$ is input (scalar or vector)
- $y$ is output (scalar or vector)
- $\mathbf{A}$ is system matrix ($n \times n$)
- $\mathbf{B}$ is input matrix ($n \times 1$)
- $\mathbf{C}$ is output matrix ($1 \times n$)
- $\mathbf{D}$ is feedthrough matrix (often 0)

\subsection*{Transfer Function from State-Space}

$$G(s) = \mathbf{C}(s\mathbf{I} - \mathbf{A})^{-1}\mathbf{B} + \mathbf{D}$$

\subsection*{Stability from State-Space}

System is stable if all eigenvalues of $\mathbf{A}$ have negative real parts:
$$\text{Re}(\lambda_i) < 0 \quad \text{for all } i$$

\subsection*{Controllability and Observability}

\textbf{Controllability Matrix:}
$$\mathcal{C} = [\mathbf{B} \quad \mathbf{AB} \quad \mathbf{A}^2\mathbf{B} \quad \cdots \quad \mathbf{A}^{n-1}\mathbf{B}]$$

System is controllable if $\text{rank}(\mathcal{C}) = n$

\textbf{Observability Matrix:}
$$\mathcal{O} = \begin{bmatrix} \mathbf{C} \\ \mathbf{CA} \\ \mathbf{CA}^2 \\ \vdots \\ \mathbf{CA}^{n-1} \end{bmatrix}$$

System is observable if $\text{rank}(\mathcal{O}) = n$

\section*{Linearization}

For nonlinear system $\dot{x} = f(x, u)$, linearize about equilibrium point $(x_0, u_0)$:

$$\Delta\dot{x} = \frac{\partial f}{\partial x}\bigg|_{x_0, u_0} \Delta x + \frac{\partial f}{\partial u}\bigg|_{x_0, u_0} \Delta u$$

where $\mathbf{A} = \frac{\partial f}{\partial x}\bigg|_{x_0, u_0}$ (Jacobian)

\section*{Quick Reference}

\textbf{Typical Second-Order System:}
- Natural frequency: $\omega_n = \sqrt{k/m}$
- Damping ratio: $\zeta = c/(2\sqrt{km})$
- Settling time: $T_s \approx 4/(\zeta\omega_n)$
- Overshoot: $M_p = e^{-\zeta\pi/\sqrt{1-\zeta^2}}$

\textbf{Stability Check:}
1. Closed-loop poles in LHP
2. Routh-Hurwitz: No sign changes
3. Nyquist: No encirclement of $-1$
4. Bode: GM $> 0$, PM $> 0$

\end{document}