\documentclass[11pt]{article}
\usepackage[T1]{fontenc}
\usepackage{lmodern}
\usepackage[margin=1in]{geometry}
\usepackage{amsmath,amssymb}
\usepackage{array}
\usepackage{booktabs}
\usepackage{graphicx}

\pagestyle{empty}
\setlength{\parindent}{0pt}
\setlength{\parskip}{8pt}

\begin{document}

\begin{center}
\LARGE{\textbf{Heat Transfer Sample Problem 1}}
\end{center}

\vspace{10pt}

This is problem 1.1 on the 2023 QE.\\

\begin{figure}[h]
    \centering
    \includegraphics[width=0.75\textwidth]{heat_transfer_sample_1_fig.png}
    \label{fig:heat_transfer}
\end{figure}

\section*{Part A}
Derive an expression for variation of the mean fluid temperature $T_m(x)$ only in terms of 
known parameters. Hint: select a differential control volume and apply the energy balance 
equation. 

Convective energy is given by the following equation:
\begin{equation}
    E = \dot{m} C_p \Delta T
\end{equation}

By conservation of energy, the total energy entering the system must equal the total energy
leabing the system. 
\begin{equation}
    E_{in} + E_{wall} = E_{out} 
\end{equation}

By using a differntial control volume, we can express the terms in the energy balance as:
\begin{gather}
    E_{in} = \dot{m} C_p T_m(x) \\
    E_{out} = \dot{m} C_p T_m(x + dx) \\
    E_{wall} = q'' \pi D dx
\end{gather}

Note for the wall energy term, we are using the heat flux $q''$ multiplied by the surface
area of the pipe over the differential length $\Delta x$. The full energy balance is then:

\begin{equation}
    \dot{m} C_p T_m(x) + q'' \pi D  dx = \dot{m} C_p T_m(x + dx)
\end{equation}

We can rearrange and solve for $\frac{dT_m}{dx}$ as follows:

\begin{gather}
    q''\pi Ddx = \dot{m}C_p T_m(x+dx) - \dot{m}C_p T_m(x) \\
    q''\pi Ddx = \dot{m}C_p \left[ T_m(x+dx) - T_m(x) \right] \\
    q''\pi Ddx = \dot{m}C_p dT_m\\
    \frac{dT_m}{dx} = \frac{q''\pi D}{\dot{m}C_p}
\end{gather}

We can then substitute $q'' = 2\ddot{q_0}(1-\frac{x}{L})$ into the equation to get:
\begin{equation}
    \frac{dT_m}{dx} = \frac{2\ddot{q_0}(1-\frac{x}{L})\pi D}{\dot{m}C_p}
\end{equation}

Then, integrating both sides from 0 to $x$ gives:
\begin{gather}
    \int_{T_{m,0}}^{T_m(x)} dT_m = \int_0^x \frac{2\ddot{q_0}(1-\frac{x}{L})\pi D}{\dot{m}C_p} dx \\
    T_m(x) - T_{m,0} = \frac{2\ddot{q_0}\pi D}{\dot{m}C_p} \left( x - \frac{x^2}{2L} \right) \\
    \boxed{T_m(x) = T_{m,0} + \frac{2\ddot{q_0}\pi D}{\dot{m}C_p} \left( x - \frac{x^2}{2L} \right)}
\end{gather}

\section*{Part B}

Derive an expression for the total rate of heat transfer q to the fluid 
only in terms of known parameters. \\

The total rate of heat transfer to the fluid can be found by integrating the heat flux
over the surface area of the pipe:

\begin{gather}
    q = \int_0^L q'' \pi D dx \\
    q = \int_0^L 2\ddot{q_0}(1-\frac{x}{L}) \pi D dx \\
    q = 2\ddot{q_0}\pi D \int_0^L (1-\frac{x}{L}) dx \\
    q = 2\ddot{q_0}\pi D \left[ x - \frac{x^2}{2L} \right]_0^L \\
    \boxed{q = \ddot{q_0}\pi DL}
\end{gather}

\section*{Part C}

Derive an expression for variation of the surface (wall) temperature $T_s(x)$
only in terms of known parameters. \\

The surface temperature can be found using the convective heat transfer equation:
\begin{equation}
    q'' = h(T_s - T_m)
\end{equation}

Rearranging for $T_s$ gives:
\begin{equation}
    T_s = \frac{q''}{h} + T_m
\end{equation}

We also know from the problem statement that the Nusselt number is:
\begin{equation}
    Nu = \frac{hD}{k} = 0.023 Re^{0.8} Pr^{0.4}
\end{equation}

Rearranging for $h$ gives:
\begin{equation}
    h = \frac{0.023 k Re^{0.8} Pr^{0.4}}{D}
\end{equation}

We can then substitute $h$, $q''$, and our expression for $T_m$ into the equation for $T_s$.
\begin{gather}
    T_s = \frac{2\ddot{q_0}(1-\frac{x}{L})}{\frac{0.023 k Re^{0.8} Pr^{0.4}}{D}} + 
    \left[ T_{m,0} + \frac{2\ddot{q_0}\pi D}{\dot{m}C_p} \left( x - \frac{x^2}{2L} \right) \right] \\
    \boxed{T_s(x) = \frac{2\ddot{q_0}(1-\frac{x}{L})D}{0.023 k Re^{0.8} Pr^{0.4}} + 
    T_{m,0} + \frac{2\ddot{q_0}\pi D}{\dot{m}C_p} \left( x - \frac{x^2}{2L} \right)}
\end{gather}

\section*{Part D}

Derive an expression for the axial location $x_max$ at which the surface (wall) 
temperature is maximum. \\

To find the location where the surface temperature is maximum, we need to take the derivative
of $T_s(x)$ and set it equal to zero.
\begin{gather}
    \frac{dT_s}{dx} = \frac{d}{dx} \left[ \frac{2\ddot{q_0}(1-\frac{x}{L})D}{0.023 k Re^{0.8} Pr^{0.4}} + 
    T_{m,0} + \frac{2\ddot{q_0}\pi D}{\dot{m}C_p} \left( x - \frac{x^2}{2L} \right) \right] \\
    \frac{dT_s}{dx} = \frac{-2\ddot{q_0}D}{0.023 k Re^{0.8} Pr^{0.4} L} + 
    \frac{2\ddot{q_0}\pi D}{\dot{m}C_p} \left( 1 - \frac{x}{L} \right)
\end{gather}

Setting the derivative equal to zero and solving for $x$ gives:
\begin{gather}
    0 = \frac{-2\ddot{q_0}D}{0.023 k Re^{0.8} Pr^{0.4} L} + 
    \frac{2\ddot{q_0}\pi D}{\dot{m}C_p} \left( 1 - \frac{x}{L} \right) \\
    \frac{2\ddot{q_0}D}{0.023 k Re^{0.8} Pr^{0.4} L} = 
    \frac{2\ddot{q_0}\pi D}{\dot{m}C_p} \left( 1 - \frac{x}{L} \right) \\
    \frac{\dot{m}C_p}{0.023 k Re^{0.8} Pr^{0.4} L \pi} = 1 - \frac{x}{L} \\
    \frac{x}{L} = 1 - \frac{\dot{m}C_p}{0.023 k Re^{0.8} Pr^{0.4} L \pi} \\
    \boxed{x_{max} = L \left( 1 - \frac{\dot{m}C_p}{0.023 k Re^{0.8} Pr^{0.4} L \pi} \right)}
\end{gather}



\end{document}