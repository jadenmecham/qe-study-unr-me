\documentclass[11pt]{article}
\usepackage[T1]{fontenc}
\usepackage{lmodern}
\usepackage[margin=1in]{geometry}
\usepackage{amsmath,amssymb}
\usepackage{array}
\usepackage{booktabs}

\pagestyle{empty}
\setlength{\parindent}{0pt}
\setlength{\parskip}{8pt}

\begin{document}

\begin{center}
\LARGE{\textbf{Solid Mechanics}}
\end{center}

\vspace{10pt}

\section*{Stress and Strain}

\subsection*{Normal Stress}

\textbf{Axial Stress:}
$$\sigma = \frac{P}{A}$$

where $P$ is axial force and $A$ is cross-sectional area

\textbf{Sign Convention:}
- Tension: positive
- Compression: negative

\subsection*{Shear Stress}

\textbf{Direct Shear:}
$$\tau = \frac{V}{A}$$

where $V$ is shear force and $A$ is area

\textbf{Torsional Shear Stress:}
$$\tau = \frac{T\rho}{J}$$

where $T$ is torque, $\rho$ is radial distance, $J$ is polar moment of inertia

For circular shaft:
$$\tau_{max} = \frac{Tr}{J} = \frac{16T}{\pi d^3}$$

\subsection*{Normal Strain}

$$\epsilon = \frac{\Delta L}{L} = \frac{dL}{L}$$

\textbf{Sign Convention:}
- Elongation: positive
- Contraction: negative

\subsection*{Shear Strain}

$$\gamma = \tan\theta \approx \theta \quad \text{(for small angles)}$$

where $\theta$ is the change in angle (in radians)

\subsection*{Hooke's Law}

\textbf{Uniaxial:}
$$\sigma = E\epsilon$$

where $E$ is Young's modulus (modulus of elasticity)

\textbf{Shear:}
$$\tau = G\gamma$$

where $G$ is shear modulus

\textbf{Relationship between E and G:}
$$G = \frac{E}{2(1 + \nu)}$$

where $\nu$ is Poisson's ratio

\subsection*{Poisson's Ratio}

$$\nu = -\frac{\epsilon_{\text{lateral}}}{\epsilon_{\text{axial}}}$$

Typical values: $0.25$ to $0.35$ for most metals

\subsection*{Generalized Hooke's Law (3D)}

$$\epsilon_x = \frac{1}{E}[\sigma_x - \nu(\sigma_y + \sigma_z)]$$
$$\epsilon_y = \frac{1}{E}[\sigma_y - \nu(\sigma_x + \sigma_z)]$$
$$\epsilon_z = \frac{1}{E}[\sigma_z - \nu(\sigma_x + \sigma_y)]$$

$$\gamma_{xy} = \frac{\tau_{xy}}{G}, \quad \gamma_{yz} = \frac{\tau_{yz}}{G}, \quad \gamma_{xz} = \frac{\tau_{xz}}{G}$$

\section*{Axial Loading}

\subsection*{Axial Deformation}

$$\delta = \frac{PL}{AE}$$

For varying load, area, or material:
$$\delta = \int_0^L \frac{P(x)}{A(x)E(x)}\,dx$$

\subsection*{Statically Indeterminate Problems}

For statically indeterminate systems:

1. Write equilibrium equations
2. Write compatibility (deformation) equations
3. Write force-deformation relationships
4. Solve system of equations

\textbf{Example - Bar with supports at both ends:}

Equilibrium: $R_A + R_B = P$

Compatibility: $\delta_A + \delta_B = 0$ (or specified gap/displacement)

\subsection*{Thermal Effects}

\textbf{Thermal Strain:}
$$\epsilon_T = \alpha \Delta T$$

where $\alpha$ is coefficient of thermal expansion and $\Delta T$ is temperature change

\textbf{Total Deformation (unconstrained):}
$$\delta_T = \alpha L \Delta T$$

\textbf{Thermal Stress (fully constrained):}
$$\sigma_T = E\alpha \Delta T$$

\textbf{Combined mechanical and thermal:}
$$\delta = \frac{PL}{AE} + \alpha L \Delta T$$

\subsection*{Factor of Safety}

$$n = \frac{\sigma_{\text{allowable}}}{\sigma_{\text{actual}}} = \frac{\sigma_{\text{fail}}}{\sigma_{\text{actual}}}$$

For design:
$$\sigma_{\text{allowable}} = \frac{\sigma_{\text{yield}}}{n}$$

Typical values: $n = 1.5$ to $3$ (higher for uncertain loading)

\section*{Torsion}

\subsection*{Angle of Twist}

$$\phi = \frac{TL}{GJ}$$

For varying torque, geometry, or material:
$$\phi = \int_0^L \frac{T(x)}{G(x)J(x)}\,dx$$

\subsection*{Polar Moment of Inertia}

\textbf{Solid Circular Shaft:}
$$J = \frac{\pi d^4}{32} = \frac{\pi r^4}{2}$$

\textbf{Hollow Circular Shaft:}
$$J = \frac{\pi(d_o^4 - d_i^4)}{32} = \frac{\pi(r_o^4 - r_i^4)}{2}$$

\textbf{Thin-Walled Tube:}
$$J \approx 2\pi r^3 t$$

where $r$ is mean radius and $t$ is wall thickness

\subsection*{Power Transmission}

$$P = T\omega = 2\pi nT$$

where $P$ is power (watts), $T$ is torque (N·m), $\omega$ is angular velocity (rad/s), and $n$ is rotational speed (rev/s)

\textbf{In imperial units:}
$$P(\text{hp}) = \frac{T(\text{lb·in}) \cdot n(\text{rpm})}{63,025}$$

\section*{Bending of Beams}

\subsection*{Flexure Formula (Bending Stress)}

$$\sigma = \frac{My}{I}$$

where:
- $M$ is bending moment at section
- $y$ is distance from neutral axis (positive for tension side)
- $I$ is second moment of area about neutral axis

\textbf{Maximum Bending Stress:}
$$\sigma_{max} = \frac{Mc}{I} = \frac{M}{S}$$

where $c$ is distance to extreme fiber and $S = I/c$ is section modulus

\subsection*{Section Modulus}

\textbf{Rectangular Section ($b \times h$):}
$$S = \frac{bh^2}{6}$$

\textbf{Circular Section (diameter $d$):}
$$S = \frac{\pi d^3}{32}$$

\textbf{Hollow Circular Section:}
$$S = \frac{\pi(d_o^4 - d_i^4)}{32d_o}$$

\subsection*{Second Moment of Area (Area Moment of Inertia)}

\textbf{Definition:}
$$I = \int y^2\,dA$$

\textbf{Parallel Axis Theorem:}
$$I_x = I_{\bar{x}} + Ad^2$$

where $I_{\bar{x}}$ is moment about centroidal axis and $d$ is distance between axes

\textbf{Common Cross-Sections:}

\textit{Rectangle ($b \times h$, about centroid):}
$$I = \frac{bh^3}{12}$$

\textit{Circle (diameter $d$, about centroid):}
$$I = \frac{\pi d^4}{64}$$

\textit{Hollow Circle (outer $d_o$, inner $d_i$):}
$$I = \frac{\pi(d_o^4 - d_i^4)}{64}$$

\textit{Triangle (base $b$, height $h$, about base):}
$$I = \frac{bh^3}{12}$$

\textit{Triangle (about centroid):}
$$I = \frac{bh^3}{36}$$

\subsection*{Beam Deflection}

\textbf{Differential Equations:}

$$\frac{d^2v}{dx^2} = \frac{M}{EI}$$

$$\frac{dv}{dx} = \theta = \int \frac{M}{EI}\,dx$$

$$v = \iint \frac{M}{EI}\,dx\,dx$$

where $v$ is deflection and $\theta$ is slope

\textbf{Sign Convention:}
- Positive $M$: causes compression on top
- Positive $v$: downward deflection
- Positive $\theta$: counterclockwise rotation

\subsection*{Common Beam Deflections}

\textbf{Simply Supported Beam - Point Load at Center:}
$$v_{max} = \frac{PL^3}{48EI}$$

\textbf{Simply Supported Beam - Uniform Load:}
$$v_{max} = \frac{5wL^4}{384EI}$$

\textbf{Cantilever Beam - Point Load at End:}
$$v_{max} = \frac{PL^3}{3EI}$$

\textbf{Cantilever Beam - Uniform Load:}
$$v_{max} = \frac{wL^4}{8EI}$$

\subsection*{Singularity Functions (Macaulay Method)}

\textbf{Unit Step Function:}
$$\langle x - a \rangle^0 = \begin{cases} 0 & x < a \\ 1 & x \geq a \end{cases}$$

\textbf{Ramp Function:}
$$\langle x - a \rangle^n = \begin{cases} 0 & x < a \\ (x-a)^n & x \geq a \end{cases}$$

\textbf{Integration:}
$$\int \langle x - a \rangle^n\,dx = \frac{\langle x - a \rangle^{n+1}}{n+1}$$

\textbf{Point Load:}

Use $\langle x - a \rangle^{-1}$ for point load at $x = a$

\section*{Shear and Moment Diagrams}

\subsection*{Relationships}

$$\frac{dV}{dx} = -w(x)$$

$$\frac{dM}{dx} = V(x)$$

$$M = \int V\,dx$$

where $w(x)$ is distributed load, $V$ is shear force, and $M$ is bending moment

\subsection*{Sign Conventions}

\textbf{Shear Force:}
- Positive: causes clockwise rotation of element

\textbf{Bending Moment:}
- Positive: causes compression on top, tension on bottom (concave up)

\subsection*{Key Points}

- $V$ changes abruptly at point loads
- $V$ varies linearly under uniform load
- $M$ has maximum/minimum where $V = 0$
- $M$ changes abruptly at concentrated moments

\section*{Transverse Shear Stress}

\subsection*{Shear Formula}

$$\tau = \frac{VQ}{Ib}$$

where:
- $V$ is shear force
- $Q = \bar{y}'A'$ is first moment of area above (or below) the point
- $I$ is second moment of area of entire cross-section
- $b$ is width at the point of interest

\textbf{Maximum Shear Stress (usually at neutral axis):}
$$\tau_{max} = \frac{VQ_{max}}{Ib}$$

\subsection*{Shear Stress in Common Sections}

\textbf{Rectangular Section:}
$$\tau_{max} = \frac{3V}{2A} = \frac{3V}{2bh}$$

\textbf{Circular Section:}
$$\tau_{max} = \frac{4V}{3A} = \frac{16V}{3\pi d^2}$$

\textbf{I-Beam (web):}
$$\tau_{web} \approx \frac{V}{A_{web}} = \frac{V}{t_w h_{web}}$$

\subsection*{Shear Flow}

$$q = \frac{VQ}{I}$$

where $q$ is shear flow (force per unit length)

Used for:
- Built-up beams (fastener spacing)
- Thin-walled sections
- Composite beams

\section*{Combined Loading}

\subsection*{Superposition}

For linear elastic materials, stresses from different loads can be superimposed:
$$\sigma_{total} = \sigma_1 + \sigma_2 + \cdots$$

\textbf{Common Combinations:}

\textit{Axial + Bending:}
$$\sigma = \frac{P}{A} \pm \frac{My}{I}$$

\textit{Bending in Two Planes:}
$$\sigma = \frac{M_x y}{I_x} + \frac{M_y x}{I_y}$$

\textit{Torsion + Bending:}
$$\tau_{max} = \sqrt{\left(\frac{Tr}{J}\right)^2 + \left(\frac{V Q}{Ib}\right)^2}$$

\section*{Stress Transformation}

\subsection*{Plane Stress}

For a state of stress $\sigma_x$, $\sigma_y$, $\tau_{xy}$, the stress on a plane at angle $\theta$:

\textbf{Normal Stress:}
$$\sigma_{x'} = \frac{\sigma_x + \sigma_y}{2} + \frac{\sigma_x - \sigma_y}{2}\cos(2\theta) + \tau_{xy}\sin(2\theta)$$

\textbf{Shear Stress:}
$$\tau_{x'y'} = -\frac{\sigma_x - \sigma_y}{2}\sin(2\theta) + \tau_{xy}\cos(2\theta)$$

\subsection*{Principal Stresses}

\textbf{Principal Stresses:}
$$\sigma_{1,2} = \frac{\sigma_x + \sigma_y}{2} \pm \sqrt{\left(\frac{\sigma_x - \sigma_y}{2}\right)^2 + \tau_{xy}^2}$$

where $\sigma_1$ is maximum and $\sigma_2$ is minimum

\textbf{Principal Angles:}
$$\tan(2\theta_p) = \frac{2\tau_{xy}}{\sigma_x - \sigma_y}$$

Two values of $\theta_p$ differ by $90°$

\textbf{Maximum Shear Stress:}
$$\tau_{max} = \sqrt{\left(\frac{\sigma_x - \sigma_y}{2}\right)^2 + \tau_{xy}^2} = \frac{\sigma_1 - \sigma_2}{2}$$

\textbf{Angle for Maximum Shear:}
$$\theta_s = \theta_p \pm 45°$$

\subsection*{Mohr's Circle}

\textbf{Center:}
$$C = \frac{\sigma_x + \sigma_y}{2}$$

\textbf{Radius:}
$$R = \sqrt{\left(\frac{\sigma_x - \sigma_y}{2}\right)^2 + \tau_{xy}^2}$$

\textbf{Construction:}
1. Plot point $A(\sigma_x, \tau_{xy})$ and $B(\sigma_y, -\tau_{xy})$
2. Draw circle with diameter $AB$
3. Principal stresses are intercepts on $\sigma$ axis
4. Maximum shear stress is radius of circle

\textbf{Sign Convention for Mohr's Circle:}
- $\sigma$: tension positive (right), compression negative (left)
- $\tau$: plot as given (up for positive, down for negative)
- Angles on circle are $2\theta$ (double physical angle)
- Rotate counterclockwise on circle for counterclockwise rotation of element

\section*{Pressure Vessels}

\subsection*{Thin-Walled Cylindrical Vessel}

Thin-walled criterion: $r/t \geq 10$

\textbf{Hoop Stress (Circumferential):}
$$\sigma_1 = \frac{pr}{t}$$

\textbf{Longitudinal Stress:}
$$\sigma_2 = \frac{pr}{2t}$$

\textbf{Maximum Shear Stress:}
$$\tau_{max} = \frac{\sigma_1 - \sigma_2}{2} = \frac{pr}{4t}$$

where $p$ is internal pressure, $r$ is mean radius, $t$ is wall thickness

\subsection*{Thin-Walled Spherical Vessel}

\textbf{Stress (same in all directions):}
$$\sigma = \frac{pr}{2t}$$

\subsection*{Thick-Walled Cylinders}

\textbf{Lamé Equations:}

$$\sigma_r = A - \frac{B}{r^2}$$

$$\sigma_\theta = A + \frac{B}{r^2}$$

where constants $A$ and $B$ are determined from boundary conditions

For internal pressure $p_i$ and external pressure $p_o = 0$:

$$\sigma_r = \frac{p_i r_i^2}{r_o^2 - r_i^2}\left(1 - \frac{r_o^2}{r^2}\right)$$

$$\sigma_\theta = \frac{p_i r_i^2}{r_o^2 - r_i^2}\left(1 + \frac{r_o^2}{r^2}\right)$$

\section*{Column Buckling}

\subsection*{Euler Buckling Load}

\textbf{Critical Load:}
$$P_{cr} = \frac{\pi^2 EI}{(KL)^2}$$

where:
- $E$ is Young's modulus
- $I$ is minimum second moment of area
- $L$ is column length
- $K$ is effective length factor

\textbf{Effective Length Factor $K$:}
- Both ends pinned: $K = 1.0$
- Both ends fixed: $K = 0.5$
- One fixed, one pinned: $K = 0.7$
- One fixed, one free: $K = 2.0$

\textbf{Critical Stress:}
$$\sigma_{cr} = \frac{P_{cr}}{A} = \frac{\pi^2 E}{(KL/r)^2}$$

where $r = \sqrt{I/A}$ is radius of gyration

\textbf{Slenderness Ratio:}
$$\frac{KL}{r}$$

Euler formula valid for slender columns (high slenderness ratio)

\subsection*{Johnson Parabola (Intermediate Columns)}

For intermediate slenderness ratios:
$$\sigma_{cr} = \sigma_y\left[1 - \frac{1}{4}\left(\frac{KL/r}{(KL/r)_c}\right)^2\right]$$

where $(KL/r)_c = \sqrt{2\pi^2 E/\sigma_y}$ is transition slenderness ratio

\subsection**{Eccentric Loading}

\textbf{Secant Formula:}
$$\frac{P}{A} = \frac{\sigma_{max}}{1 + (ec/r^2)\sec[(L/2r)\sqrt{P/(EA)}]}$$

where $e$ is eccentricity of load

\section*{Energy Methods}

\subsection*{Strain Energy}

\textbf{Axial Loading:}
$$U = \int_0^L \frac{P^2}{2AE}\,dx = \frac{P^2L}{2AE}$$

\textbf{Torsion:}
$$U = \int_0^L \frac{T^2}{2GJ}\,dx = \frac{T^2L}{2GJ}$$

\textbf{Bending:}
$$U = \int_0^L \frac{M^2}{2EI}\,dx$$

\textbf{Shear:}
$$U = \int_0^L \frac{V^2}{2GA}\,dx$$

\textbf{Total Strain Energy:}
$$U = U_{axial} + U_{torsion} + U_{bending} + U_{shear}$$

\subsection*{Castigliano's Theorem}

\textbf{First Theorem (displacement):}
$$\delta_i = \frac{\partial U}{\partial P_i}$$

where $\delta_i$ is displacement at point where force $P_i$ is applied

\textbf{Second Theorem (force):}
$$P_i = \frac{\partial U}{\partial \delta_i}$$

\textbf{For finding deflections:}

If no force at desired location, apply dummy load $Q$:
$$\delta = \frac{\partial U}{\partial Q}\bigg|_{Q=0}$$

\subsection*{Principle of Virtual Work}

$$\sum P_i \delta_i = \int_0^L \frac{M\,\delta M}{EI}\,dx$$

Used for deflection calculations in complex structures

\section*{Composite Beams}

\textbf{Transformed Section Method:}

For beam with two materials (1 and 2):

Transform material 2 to equivalent material 1:
$$n = \frac{E_2}{E_1}$$

Width of material 2 becomes: $b_{2,\text{equiv}} = nb_2$

Solve as single-material beam with transformed geometry

\textbf{Stress in Each Material:}
$$\sigma_1 = \frac{My}{I_{transformed}}$$
$$\sigma_2 = n\frac{My}{I_{transformed}}$$

\section*{Curved Beams}

For beams with significant initial curvature:

\textbf{Curved Beam Formula:}
$$\sigma = \frac{M(R - r)}{Aer}$$

where:
- $M$ is bending moment
- $R$ is radius to neutral axis
- $r$ is radius to point of interest
- $A$ is cross-sectional area
- $e = R - r_c$ is eccentricity ($r_c$ is radius to centroid)

\section*{Stress Concentrations}

\textbf{Stress Concentration Factor:}
$$K_t = \frac{\sigma_{max}}{\sigma_{nominal}}$$

Common sources:
- Holes
- Notches
- Fillets
- Changes in cross-section
- Keyways

For static loading with ductile materials, $K_t$ can often be ignored (yielding redistributes stress)

For fatigue or brittle materials, $K_t$ is critical

\section*{Material Properties}

\textbf{Common Values:}

\begin{center}
\begin{tabular}{lccc}
\toprule
\textbf{Material} & $E$ (GPa) & $G$ (GPa) & $\nu$ \\
\midrule
Steel & 200 & 80 & 0.30 \\
Aluminum & 70 & 26 & 0.33 \\
Concrete & 25-30 & - & 0.15-0.20 \\
Timber & 10-15 & - & - \\
\bottomrule
\end{tabular}
\end{center}



\usepackage{amsmath} % Add this to your preamble

\section*{Thick-Walled Cylinder (Lamé's Equations)}

For a thick-walled cylinder with internal radius $r_i$ and external radius $r_o$, the stresses at any radial position $r$ are given by:

\begin{align}
    \sigma_r &= \frac{P_i r_i^2 - P_o r_o^2}{r_o^2 - r_i^2} - \frac{r_i^2 r_o^2 (P_i - P_o)}{r^2(r_o^2 - r_i^2)} \\
    \sigma_\theta &= \frac{P_i r_i^2 - P_o r_o^2}{r_o^2 - r_i^2} + \frac{r_i^2 r_o^2 (P_i - P_o)}{r^2(r_o^2 - r_i^2)}
\end{align}

\textbf{Special Case: Internal Pressure Only ($P_o = 0$)}
\begin{equation}
    \sigma_{\theta} = \frac{P_i r_i^2}{r_o^2 - r_i^2} \left( 1 + \frac{r_o^2}{r^2} \right), \quad 
    \sigma_{r} = \frac{P_i r_i^2}{r_o^2 - r_i^2} \left( 1 - \frac{r_o^2}{r^2} \right)
\end{equation}

\textbf{Axial Stress ($\sigma_z$):}
\begin{itemize}
    \item \textbf{Open Ends:} $\sigma_z = 0$
    \item \textbf{Closed Ends:} $\sigma_z = \frac{P_i r_i^2 - P_o r_o^2}{r_o^2 - r_i^2}$ (constant throughout)
\end{itemize}


\end{document}
