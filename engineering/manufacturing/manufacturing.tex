\documentclass[11pt]{article}
\usepackage[T1]{fontenc}
\usepackage{lmodern}
\usepackage[margin=1in]{geometry}
\usepackage{amsmath,amssymb}
\usepackage{array}
\usepackage{booktabs}

\pagestyle{empty}
\setlength{\parindent}{0pt}
\setlength{\parskip}{8pt}

\begin{document}

\begin{center}
\LARGE{\textbf{Manufacturing}}
\end{center}

\vspace{10pt}

\section*{Metal Casting}

\subsection*{Solidification and Cooling}

\textbf{Chvorinov's Rule:}
$$t = C\left(\frac{V}{A}\right)^n$$

where:
- $t$ = total solidification time
- $V$ = volume of casting
- $A$ = surface area of casting
- $C$ = mold constant (depends on material and mold properties)
- $n$ = exponent (typically $n = 2$)

\textbf{Modulus of Casting:}
$$M = \frac{V}{A}$$

Larger modulus → slower cooling → larger grain size

\textbf{Riser Design:}

For riser to solidify last:
$$\left(\frac{V}{A}\right)_{\text{riser}} > \left(\frac{V}{A}\right)_{\text{casting}}$$

\subsection*{Fluidity and Filling}

\textbf{Fluidity:} Ability of molten metal to flow and fill mold cavities

Factors affecting fluidity:
- Temperature (higher → better fluidity)
- Composition (lower melting point → better fluidity)
- Surface tension
- Oxide formation

\textbf{Pouring Basin and Sprue:}

Flow rate through sprue:
$$Q = A_2v_2 = A_2\sqrt{2gh}$$

where $h$ is height of molten metal above sprue base

\subsection*{Shrinkage}

\textbf{Total Shrinkage = Liquid shrinkage + Solidification shrinkage + Solid shrinkage}

Typical total shrinkage: 3-8\% depending on material

Must account for shrinkage in pattern making

\section*{Metal Forming}

\subsection*{True Stress and True Strain}

\textbf{True Strain:}
$$\epsilon_T = \ln\left(\frac{L}{L_0}\right) = \ln(1 + \epsilon_E)$$

where $\epsilon_E$ is engineering strain

\textbf{True Stress:}
$$\sigma_T = \sigma_E(1 + \epsilon_E)$$

For plastic deformation (constant volume):
$$\sigma_T = \frac{F}{A} = \frac{FL}{A_0L_0}$$

\subsection*{Flow Stress}

\textbf{Flow Curve (Power Law):}
$$\bar{\sigma} = K\epsilon^n$$

where:
- $\bar{\sigma}$ = flow stress (true stress)
- $K$ = strength coefficient
- $\epsilon$ = true strain
- $n$ = strain hardening exponent

\textbf{Average Flow Stress:}
$$\bar{Y}_f = \frac{K\epsilon^n}{1 + n}$$

Used to calculate forces in forming operations

\subsection*{Rolling}

\textbf{Draft:}
$$d = t_0 - t_f$$

where $t_0$ is initial thickness, $t_f$ is final thickness

\textbf{Reduction:}
$$r = \frac{d}{t_0} = \frac{t_0 - t_f}{t_0}$$

\textbf{True Strain:}
$$\epsilon = \ln\left(\frac{t_0}{t_f}\right)$$

\textbf{Roll Force (approximate):}
$$F = \bar{Y}_f wL$$

where:
- $\bar{Y}_f$ = average flow stress
- $w$ = width of strip
- $L$ = contact length $\approx \sqrt{R \cdot d}$
- $R$ = roll radius

\textbf{Roll Torque:}
$$T = 0.5FL$$

\textbf{Roll Power:}
$$P = 2\pi NT$$

where $N$ is rotational speed (rev/time)

\subsection*{Forging}

\textbf{Forging Force (open die, no friction):}
$$F = \bar{Y}_f A_f$$

where $A_f$ is final area

\textbf{With Friction (disk approximation):}
$$F = \bar{Y}_f A_f\left(1 + \frac{2\mu r}{3h}\right)$$

where:
- $\mu$ = coefficient of friction
- $r$ = radius of workpiece
- $h$ = height of workpiece

\subsection*{Extrusion}

\textbf{Extrusion Ratio:}
$$r_x = \frac{A_0}{A_f}$$

where $A_0$ is initial billet area, $A_f$ is final extrudate area

\textbf{True Strain:}
$$\epsilon = \ln(r_x)$$

\textbf{Extrusion Force (Johnson Formula):}
$$F = A_0 \bar{Y}_f \left[a + b\ln(r_x)\right]$$

where:
- $a$ = 0.8 (typical)
- $b$ = 1.2 to 1.5 (depends on friction and die angle)

\textbf{Ram Pressure:}
$$p = \frac{F}{A_0}$$

\textbf{Maximum Extrusion Ratio:}

Limited by:
- Material strength
- Press capacity
- Buckling of billet

\subsection*{Wire and Tube Drawing}

\textbf{Drawing Stress (ideal, no friction):}
$$\sigma_d = \bar{Y}_f \ln\left(\frac{A_0}{A_f}\right)$$

\textbf{With Friction:}
$$\sigma_d = \bar{Y}_f \left[1 + \frac{\mu}{\tan\alpha}\right] \ln\left(\frac{A_0}{A_f}\right)$$

where $\alpha$ is die semi-angle

\textbf{Drawing Force:}
$$F = \sigma_d A_f$$

\textbf{Drawing Power:}
$$P = Fv$$

where $v$ is drawing velocity

\textbf{Maximum Reduction per Pass:}

Limited by tensile strength of material:
$$\sigma_d \leq \sigma_{UTS}$$

\subsection*{Sheet Metal Working}

\textbf{Bend Allowance:}
$$L_b = \alpha(R + kt)$$

where:
- $\alpha$ = bend angle (radians)
- $R$ = bend radius
- $t$ = sheet thickness
- $k$ = factor (0.33 for $R < 2t$, 0.5 for $R > 2t$)

\textbf{Minimum Bend Radius:}
$$R_{min} = \frac{t}{2}\left(\frac{100}{\%RA} - 1\right)$$

where $\%RA$ is percent reduction in area at fracture

\textbf{Bending Force:}
$$F = \frac{K_{bf}TS_{ut}w^2}{D}$$

where:
- $K_{bf}$ = bending factor (depends on die geometry)
- $TS$ = tensile strength
- $w$ = width
- $D$ = die opening width

\textbf{Deep Drawing:}

Drawing ratio:
$$DR = \frac{D_0}{D_p}$$

where $D_0$ is blank diameter, $D_p$ is punch diameter

Maximum $DR \approx 2.0$ for single draw

\textbf{Limiting Drawing Ratio (LDR):}
$$LDR = \frac{D_{0,max}}{D_p}$$

\textbf{Drawing Force:}
$$F = \pi D_p t \bar{Y}_f (DR - 0.7)$$

\section*{Machining}

\subsection*{Cutting Speed, Feed, and Depth of Cut}

\textbf{Cutting Speed:}
$$v = \frac{\pi Dn}{1000}$$

where:
- $v$ = cutting speed (m/min)
- $D$ = workpiece diameter (mm)
- $n$ = rotational speed (rpm)

\textbf{Feed:}
$$f = n \cdot f_r$$

where $f_r$ is feed per revolution (mm/rev)

\textbf{Material Removal Rate (MRR):}
$$MRR = v \cdot f \cdot d$$

where $d$ is depth of cut

For turning:
$$MRR = \frac{\pi Dnf_rd}{1000}$$

\subsection*{Cutting Forces and Power}

\textbf{Cutting Force:}
$$F_c = K_s \cdot A_c$$

where:
- $K_s$ = specific cutting energy (material property)
- $A_c$ = cross-sectional area of cut = $f \times d$

\textbf{Power:}
$$P_c = F_c \cdot v$$

\textbf{Specific Cutting Energy:}

Varies with material:
- Aluminum: 0.4-1.1 GPa
- Steel: 2.7-9.3 GPa
- Titanium: 3.0-4.1 GPa

\subsection*{Tool Life}

\textbf{Taylor Tool Life Equation:}
$$vT^n = C$$

or

$$T = \frac{C}{v^{1/n}}$$

where:
- $T$ = tool life (min)
- $v$ = cutting speed (m/min)
- $n$ = exponent (typically 0.1-0.5, 0.125 for HSS, 0.25-0.4 for carbide)
- $C$ = constant (depends on material, tool, conditions)

\textbf{Extended Tool Life Equation:}
$$vT^n f^m d^p = C$$

\textbf{Cost per Part:}

Optimal cutting speed minimizes:
$$C_{part} = C_{machine}t_m + \frac{C_{tool} + C_{change}}{n_{parts}}$$

where $t_m$ is machining time

\subsection*{Turning}

\textbf{Machining Time:}
$$t_m = \frac{L}{f_r n} = \frac{L}{f_r} \cdot \frac{1000}{\pi Dn}$$

where $L$ is length of cut

\textbf{For facing operation:}
$$t_m = \frac{r}{f_r n}$$

where $r$ is radius

\subsection*{Milling}

\textbf{Cutting Speed:}
$$v = \frac{\pi Dn}{1000}$$

where $D$ is cutter diameter

\textbf{Feed per Tooth:}
$$f_t = \frac{f}{n \cdot N_t}$$

where:
- $f$ = table feed rate (mm/min)
- $N_t$ = number of teeth on cutter

\textbf{Material Removal Rate:}
$$MRR = w \cdot d \cdot f$$

where $w$ is width of cut

\textbf{Machining Time:}
$$t_m = \frac{L + L_a}{f}$$

where $L_a$ is approach distance

\subsection*{Drilling}

\textbf{Feed:}
$$f = nf_r$$

where $f_r$ is feed per revolution

\textbf{Material Removal Rate:}
$$MRR = \frac{\pi D^2}{4} \cdot f$$

\textbf{Drilling Time:}
$$t = \frac{L + A}{f_r n}$$

where $A$ is approach allowance (typically $A = D/(2\tan\theta)$ for drill point angle $2\theta$)

\textbf{Torque:}
$$T = \frac{1}{2}F_c \cdot \frac{D}{2}$$

\textbf{Power:}
$$P = \frac{2\pi nT}{60}$$

\section*{Surface Finish and Metrology}

\subsection*{Surface Roughness}

\textbf{Average Roughness ($R_a$):}
$$R_a = \frac{1}{L}\int_0^L |y(x)|\,dx$$

Arithmetic average of absolute deviations from mean line

\textbf{Root Mean Square Roughness ($R_q$):}
$$R_q = \sqrt{\frac{1}{L}\int_0^L y^2(x)\,dx}$$

\textbf{Maximum Peak-to-Valley Height ($R_t$):}

Distance between highest peak and lowest valley

\textbf{Theoretical Surface Roughness (turning, shaping):}
$$R_t = \frac{f^2}{8R_n}$$

where:
- $f$ = feed
- $R_n$ = tool nose radius

\subsection*{Geometric Dimensioning and Tolerancing (GD\&T)}

\textbf{Fundamental Tolerance Equation:}
$$\text{Tolerance} = \text{Upper Limit} - \text{Lower Limit}$$

\textbf{Bilateral Tolerance:}
$$50.0 \pm 0.1 \text{ mm}$$

\textbf{Unilateral Tolerance:}
$$50.0^{+0.2}_{0} \text{ mm}$$

\textbf{Fits:}

- Clearance fit: $\text{Minimum hole size} > \text{Maximum shaft size}$
- Interference fit: $\text{Maximum hole size} < \text{Minimum shaft size}$
- Transition fit: Can be either clearance or interference

\textbf{Standard Tolerance Grades (IT):}

IT01 to IT18, where lower numbers indicate tighter tolerances

\section*{Non-Traditional Machining}

\subsection*{Electrical Discharge Machining (EDM)}

\textbf{Material Removal Rate:}
$$MRR = K_m \cdot I \cdot V$$

where:
- $I$ = discharge current
- $V$ = gap voltage
- $K_m$ = material removal constant

Advantages: Can machine hard materials, complex shapes

\subsection*{Electrochemical Machining (ECM)}

\textbf{Material Removal Rate (Faraday's Law):}
$$MRR = \frac{CIA}{nF\rho}$$

where:
- $C$ = current efficiency
- $I$ = current
- $A$ = atomic weight
- $n$ = valence
- $F$ = Faraday's constant (96,485 C/mol)
- $\rho$ = density

\subsection*{Laser Beam Machining (LBM)}

\textbf{Energy Density:}
$$E_d = \frac{P}{A}$$

where $P$ is laser power and $A$ is spot area

High energy density → vaporization of material

\section*{Additive Manufacturing (3D Printing)}

\subsection*{Build Time Estimation}

\textbf{Layer-by-Layer Build:}
$$t = \frac{h}{v_b}$$

where:
- $h$ = total build height
- $v_b$ = build rate (height/time)

\textbf{Material Usage:}
$$m = V \cdot \rho$$

where $V$ is part volume and $\rho$ is material density

\subsection*{Common AM Processes}

- Fused Deposition Modeling (FDM)
- Stereolithography (SLA)
- Selective Laser Sintering (SLS)
- Electron Beam Melting (EBM)
- Binder Jetting

\section*{Heat Treatment}

\subsection*{Time-Temperature Transformation}

\textbf{Cooling Rate:}
$$CR = \frac{\Delta T}{\Delta t}$$

Different cooling rates produce different microstructures:
- Slow cool: Pearlite
- Medium cool: Bainite
- Fast cool (quench): Martensite

\subsection*{Hardness After Heat Treatment}

\textbf{Jominy End-Quench Test:}

Measures hardenability by cooling rate

Distance from quenched end correlates with cooling rate

\textbf{Tempering:}

Reduces hardness and increases toughness

Temperature and time determine final properties

\section*{TRIBOLOGY}

\section*{Friction}

\subsection*{Coulomb (Dry) Friction}

\textbf{Friction Force:}
$$F_f = \mu N$$

where:
- $\mu$ = coefficient of friction
- $N$ = normal force

\textbf{Static Friction:}
$$F_s \leq \mu_s N$$

\textbf{Kinetic Friction:}
$$F_k = \mu_k N$$

Typically $\mu_k < \mu_s$

\subsection*{Friction Laws}

\textbf{Amontons' Laws:}
1. Friction force proportional to normal load
2. Friction force independent of apparent contact area
3. Kinetic friction independent of sliding velocity (approximately)

\subsection*{Friction Mechanisms}

\textbf{Adhesion:} Bonding at contact points, must be sheared

\textbf{Plowing:} Harder asperities plow through softer material

\textbf{Total Friction:}
$$\mu = \mu_{adhesion} + \mu_{plowing}$$

\subsection*{Rolling Friction}

\textbf{Rolling Resistance Coefficient:}
$$\mu_r = \frac{F_r}{N}$$

Generally $\mu_r \ll \mu_k$

\textbf{Rolling Resistance Force:}
$$F_r = \frac{C_r N}{r}$$

where:
- $C_r$ = rolling resistance coefficient (length)
- $r$ = wheel radius

\section*{Wear}

\subsection*{Archard Wear Equation}

\textbf{Wear Volume:}
$$V = K\frac{NL}{H}$$

where:
- $K$ = dimensionless wear coefficient
- $N$ = normal load
- $L$ = sliding distance
- $H$ = hardness of softer material

\textbf{Wear Rate:}
$$\frac{dV}{dt} = K\frac{Nv}{H}$$

where $v$ is sliding velocity

\textbf{Wear Coefficient:}

Ranges from $10^{-8}$ (mild wear) to $10^{-2}$ (severe wear)

\subsection*{Types of Wear}

\textbf{Adhesive Wear:} Material transfer between surfaces

\textbf{Abrasive Wear:} Hard particles or asperities scratch surface

- Two-body: Hard surface against soft
- Three-body: Hard particles between surfaces

\textbf{Fatigue Wear:} Repeated loading causes surface cracks, spalling

\textbf{Corrosive Wear:} Chemical reaction forms oxide, then removed

\textbf{Fretting Wear:} Small amplitude oscillatory motion

\subsection*{Specific Wear Rate}

$$k = \frac{V}{NL}$$

Units: mm³/(N·m)

Lower $k$ indicates better wear resistance

\section*{Lubrication}

\subsection*{Viscosity}

\textbf{Dynamic Viscosity ($\mu$):}

$$\tau = \mu\frac{du}{dy}$$

Units: Pa·s or N·s/m² or poise (1 poise = 0.1 Pa·s)

\textbf{Kinematic Viscosity ($\nu$):}
$$\nu = \frac{\mu}{\rho}$$

Units: m$^2$/s or stoke (1 stoke = $10^{-4}$ m$^2$/s)

\textbf{Temperature Dependence:}

Viscosity decreases with increasing temperature

\subsection*{Lubrication Regimes}

\textbf{Stribeck Curve:} Friction vs. $\frac{\mu v}{P}$

\textbf{Boundary Lubrication:}
- High load, low speed, thin film
- Metal-to-metal contact
- High friction ($\mu \approx 0.1$ to $0.15$)

\textbf{Mixed Lubrication:}
- Partial fluid film, partial contact
- Transition regime

\textbf{Hydrodynamic (Fluid Film) Lubrication:}
- Surfaces completely separated by fluid film
- Low friction ($\mu \approx 0.001$ to $0.01$)
- Load supported by pressure in fluid

\textbf{Elastohydrodynamic (EHL) Lubrication:}
- High pressure deforms surfaces
- Common in gears, rolling bearings

\subsection*{Reynolds Equation}

\textbf{1D Simplified:}
$$\frac{d}{dx}\left(h^3\frac{dp}{dx}\right) = 6\mu U\frac{dh}{dx}$$

where:
- $h$ = film thickness
- $p$ = pressure
- $U$ = velocity

Describes pressure distribution in fluid film

\subsection*{Minimum Film Thickness}

\textbf{For journal bearing:}
$$h_{min} = c(1 - \epsilon)$$

where:
- $c$ = radial clearance
- $\epsilon$ = eccentricity ratio

For safe operation: $h_{min} > 3\sigma$ (where $\sigma$ is composite surface roughness)

\subsection*{Lubricant Selection}

\textbf{Viscosity Index (VI):}

Measure of viscosity change with temperature

Higher VI → less change with temperature

\textbf{Pour Point:} Lowest temperature at which oil flows

\textbf{Flash Point:} Temperature at which vapor ignites

\section*{Surface Characterization}

\subsection*{Surface Topography}

\textbf{Peak Count:} Number of peaks per unit length

\textbf{Bearing Ratio:} Fraction of surface above a certain depth

\textbf{Wavelength:} Distance between repeating features

\subsection*{Contact Mechanics}

\textbf{Hertzian Contact (elastic):}

For cylinder on cylinder:
$$p_{max} = \frac{2P}{\pi bL}$$

where:
- $P$ = normal load
- $b$ = contact width
- $L$ = contact length

\textbf{Real Contact Area:}
$$A_r = \frac{N}{H}$$

where $H$ is hardness

Real area $\ll$ Apparent area

\section*{Solid Lubricants}

Common solid lubricants:
- Graphite (requires moisture or gases)
- MoS$_2$ (works in vacuum)
- PTFE (Teflon)
- Soft metals (lead, indium)

Used when liquid lubricants fail (high temp, vacuum, contamination concerns)

\section*{Quick Reference Values}

\textbf{Typical Friction Coefficients:}

\begin{center}
\begin{tabular}{lc}
\toprule
\textbf{Interface} & $\mu$ \\
\midrule
Steel on steel (dry) & 0.6-0.8 \\
Steel on steel (lubricated) & 0.05-0.1 \\
Steel on bronze (dry) & 0.2 \\
Brake materials & 0.3-0.5 \\
Rubber on pavement & 0.6-0.9 \\
Teflon on steel & 0.04-0.1 \\
\bottomrule
\end{tabular}
\end{center}

\end{document}