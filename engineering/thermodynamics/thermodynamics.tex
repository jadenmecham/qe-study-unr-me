\documentclass[11pt]{article}
\usepackage[T1]{fontenc}
\usepackage{lmodern}
\usepackage[margin=1in]{geometry}
\usepackage{amsmath,amssymb}
\usepackage{array}
\usepackage{booktabs}

\pagestyle{empty}
\setlength{\parindent}{0pt}
\setlength{\parskip}{8pt}

\begin{document}

\begin{center}
\LARGE{\textbf{Thermodynamics}}
\end{center}

\vspace{10pt}

\section*{Fundamental Concepts}

\subsection*{Properties and State}

\textbf{Intensive Properties:} Independent of mass (T, P, $\rho$, v)

\textbf{Extensive Properties:} Depend on mass (V, m, E, H, S)

\textbf{Specific Property:}
$$\text{specific property} = \frac{\text{extensive property}}{\text{mass}}$$

Example: $v = V/m$ (specific volume)

\subsection*{Pressure}

\textbf{Absolute Pressure:}
$$P_{abs} = P_{atm} + P_{gage}$$

\textbf{Vacuum Pressure:}
$$P_{vac} = P_{atm} - P_{abs}$$

Standard atmospheric pressure: $P_{atm} = 101.325$ kPa = 1 atm = 14.7 psi

\subsection*{Temperature}

\textbf{Conversions:}
$$T(K) = T(°C) + 273.15$$
$$T(R) = T(°F) + 459.67$$
$$T(°F) = \frac{9}{5}T(°C) + 32$$

\section*{Properties of Pure Substances}

\subsection*{Phase Diagram and Saturation}

\textbf{Saturation Temperature:} $T_{sat}$ at given pressure

\textbf{Saturation Pressure:} $P_{sat}$ at given temperature

\textbf{Quality (for two-phase mixture):}
$$x = \frac{m_{vapor}}{m_{total}} = \frac{m_g}{m_f + m_g}$$

\textbf{Properties in Two-Phase Region:}
$$v = v_f + x \cdot v_{fg}$$
$$u = u_f + x \cdot u_{fg}$$
$$h = h_f + x \cdot h_{fg}$$
$$s = s_f + x \cdot s_{fg}$$

where subscript $f$ denotes saturated liquid, $g$ denotes saturated vapor, and $fg$ denotes difference ($v_{fg} = v_g - v_f$)

\subsection*{Ideal Gas Law}

$$PV = mRT$$

or

$$Pv = RT$$

where $R$ is specific gas constant

\textbf{Universal Gas Constant:}
$$R_u = 8.314 \text{ kJ/(kmol·K)} = 1.986 \text{ Btu/(lbmol·R)}$$

\textbf{Specific Gas Constant:}
$$R = \frac{R_u}{M}$$

where $M$ is molar mass

\textbf{Ideal Gas Relations:}
$$\frac{P_1V_1}{T_1} = \frac{P_2V_2}{T_2}$$

\section*{First Law of Thermodynamics}

\subsection*{Energy Balance}

\textbf{General Form:}
$$\Delta E_{system} = E_{in} - E_{out}$$

\textbf{Total Energy:}
$$E = U + KE + PE = U + \frac{1}{2}mv^2 + mgz$$

\textbf{Per Unit Mass:}
$$e = u + \frac{v^2}{2} + gz$$

\subsection*{Closed System}

\textbf{Energy Balance:}
$$Q - W = \Delta U + \Delta KE + \Delta PE$$

For stationary system (negligible KE, PE):
$$Q - W = \Delta U = m(u_2 - u_1)$$

\textbf{Boundary Work (moving boundary):}
$$W_b = \int_{V_1}^{V_2} P \, dV$$

For constant pressure (isobaric):
$$W_b = P(V_2 - V_1)$$

For polytropic process ($Pv^n = C$):
$$W_b = \frac{P_2V_2 - P_1V_1}{1 - n} = \frac{mR(T_2 - T_1)}{1 - n} \quad (n \neq 1)$$

For isothermal process of ideal gas:
$$W_b = P_1V_1\ln\left(\frac{V_2}{V_1}\right) = mRT\ln\left(\frac{V_2}{V_1}\right)$$

\subsection*{Open System (Control Volume)}

\textbf{Mass Balance (Conservation of Mass):}
$$\frac{dm_{cv}}{dt} = \sum \dot{m}_{in} - \sum \dot{m}_{out}$$

Steady flow: $\sum \dot{m}_{in} = \sum \dot{m}_{out}$

\textbf{Energy Balance:}
$$\frac{dE_{cv}}{dt} = \dot{Q}_{cv} - \dot{W}_{cv} + \sum \dot{m}_{in}\left(h + \frac{v^2}{2} + gz\right)_{in} - \sum \dot{m}_{out}\left(h + \frac{v^2}{2} + gz\right)_{out}$$

\textbf{Steady-Flow Energy Equation (SFEE):}
$$\dot{Q} - \dot{W} = \sum \dot{m}_{out}\left(h + \frac{v^2}{2} + gz\right)_{out} - \sum \dot{m}_{in}\left(h + \frac{v^2}{2} + gz\right)_{in}$$

For single inlet/outlet:
$$\dot{Q} - \dot{W} = \dot{m}\left[\left(h_2 - h_1\right) + \frac{v_2^2 - v_1^2}{2} + g(z_2 - z_1)\right]$$

\textbf{Enthalpy:}
$$H = U + PV$$
$$h = u + Pv$$

\section*{Specific Heats}

\textbf{Constant Volume Specific Heat:}
$$c_v = \left(\frac{\partial u}{\partial T}\right)_v$$

\textbf{Constant Pressure Specific Heat:}
$$c_p = \left(\frac{\partial h}{\partial T}\right)_p$$

\textbf{Ideal Gas Relations:}
$$c_p = c_v + R$$

$$k = \frac{c_p}{c_v}$$

\textbf{Internal Energy Change (Ideal Gas):}
$$\Delta u = c_v \Delta T$$

\textbf{Enthalpy Change (Ideal Gas):}
$$\Delta h = c_p \Delta T$$

\section*{Second Law of Thermodynamics}

\subsection*{Kelvin-Planck Statement}

It is impossible for any device that operates on a cycle to receive heat from a single reservoir and produce net work.

\subsection*{Clausius Statement}

It is impossible to construct a device that operates in a cycle and produces no effect other than the transfer of heat from a lower-temperature body to a higher-temperature body.

\subsection*{Carnot Principles}

1. No heat engine can be more efficient than a reversible engine operating between the same two reservoirs

2. All reversible heat engines operating between the same two reservoirs have the same efficiency

\subsection*{Entropy}

\textbf{Definition:}
$$dS = \left(\frac{\delta Q}{T}\right)_{rev}$$

\textbf{Entropy Change:}
$$\Delta S = \int_1^2 \left(\frac{\delta Q}{T}\right)_{rev}$$

\textbf{Clausius Inequality:}
$$\oint \frac{\delta Q}{T} \leq 0$$

Equality for reversible process, inequality for irreversible

\textbf{Entropy Balance (Control Volume):}
$$\frac{dS_{cv}}{dt} = \sum \frac{\dot{Q}_k}{T_k} + \sum \dot{m}_i s_i - \sum \dot{m}_e s_e + \dot{S}_{gen}$$

where $\dot{S}_{gen} \geq 0$ (entropy generation)

\textbf{Isentropic Process:}
$$\Delta s = 0$$ (reversible and adiabatic)

\subsection*{Entropy Change for Different Processes}

\textbf{Ideal Gas:}
$$\Delta s = c_p\ln\left(\frac{T_2}{T_1}\right) - R\ln\left(\frac{P_2}{P_1}\right)$$

$$\Delta s = c_v\ln\left(\frac{T_2}{T_1}\right) + R\ln\left(\frac{v_2}{v_1}\right)$$

\textbf{Incompressible Substance:}
$$\Delta s = c\ln\left(\frac{T_2}{T_1}\right)$$

\textbf{Isentropic Relations (Ideal Gas):}
$$\frac{T_2}{T_1} = \left(\frac{P_2}{P_1}\right)^{(k-1)/k} = \left(\frac{v_1}{v_2}\right)^{k-1}$$

$$\frac{P_2}{P_1} = \left(\frac{v_1}{v_2}\right)^k$$

\section*{Exergy (Availability)}

\textbf{Exergy (Flow):}
$$\psi = (h - h_0) - T_0(s - s_0) + \frac{v^2}{2} + gz$$

\textbf{Exergy (Non-Flow):}
$$\phi = (u - u_0) + P_0(v - v_0) - T_0(s - s_0) + \frac{v^2}{2} + gz$$

\textbf{Exergy Destruction:}
$$X_{dest} = T_0 S_{gen}$$

\textbf{Second Law Efficiency:}
$$\eta_{II} = \frac{\text{Exergy recovered}}{\text{Exergy supplied}}$$

\section*{Power Cycles}

\subsection*{Cycle Efficiency and COP}

\textbf{Thermal Efficiency (Heat Engine):}
$$\eta_{th} = \frac{W_{net}}{Q_{in}} = 1 - \frac{Q_{out}}{Q_{in}}$$

\textbf{Coefficient of Performance (Refrigerator):}
$$COP_R = \frac{Q_L}{W_{net}} = \frac{Q_L}{Q_H - Q_L}$$

\textbf{Coefficient of Performance (Heat Pump):}
$$COP_{HP} = \frac{Q_H}{W_{net}} = \frac{Q_H}{Q_H - Q_L}$$

Note: $COP_{HP} = COP_R + 1$

\subsection*{Carnot Cycle}

\textbf{Carnot Efficiency:}
$$\eta_{Carnot} = 1 - \frac{T_L}{T_H}$$

\textbf{Carnot COP (Refrigerator):}
$$COP_{Carnot,R} = \frac{T_L}{T_H - T_L}$$

\textbf{Carnot COP (Heat Pump):}
$$COP_{Carnot,HP} = \frac{T_H}{T_H - T_L}$$

\subsection*{Air-Standard Cycles}

\textbf{Otto Cycle (Spark Ignition):}

Compression ratio: $r = v_1/v_2$

Efficiency:
$$\eta_{Otto} = 1 - \frac{1}{r^{k-1}}$$

\textbf{Diesel Cycle (Compression Ignition):}

Compression ratio: $r = v_1/v_2$

Cutoff ratio: $r_c = v_3/v_2$

Efficiency:
$$\eta_{Diesel} = 1 - \frac{1}{r^{k-1}}\left[\frac{r_c^k - 1}{k(r_c - 1)}\right]$$

\textbf{Brayton Cycle (Gas Turbine):}

Pressure ratio: $r_p = P_2/P_1$

Efficiency:
$$\eta_{Brayton} = 1 - \frac{1}{r_p^{(k-1)/k}}$$

With regeneration:
$$\eta_{regen} = 1 - \left(\frac{T_1}{T_3}\right)\left(\frac{1}{r_p^{(k-1)/k}}\right)$$

\textbf{Stirling and Ericsson Cycles:}

Both have Carnot efficiency: $\eta = 1 - T_L/T_H$

\subsection*{Rankine Cycle (Vapor Power)}

\textbf{Basic Components:}
1. Pump (1→2): $w_p = v_1(P_2 - P_1) \approx v_f(P_2 - P_1)$
2. Boiler (2→3): $q_{in} = h_3 - h_2$
3. Turbine (3→4): $w_t = h_3 - h_4$
4. Condenser (4→1): $q_{out} = h_4 - h_1$

\textbf{Efficiency:}
$$\eta_{Rankine} = \frac{w_t - w_p}{q_{in}} = \frac{(h_3 - h_4) - (h_2 - h_1)}{h_3 - h_2}$$

\textbf{With Reheat:}

Turbine work increases, efficiency increases

\textbf{With Regeneration:}

Extract steam from turbine to preheat feedwater

Increases efficiency by reducing $q_{in}$

\textbf{Back Work Ratio:}
$$bwr = \frac{w_p}{w_t}$$

Typically very small for vapor cycles (1-2\%)

\subsection*{Refrigeration Cycles}

\textbf{Ideal Vapor-Compression Cycle:}

1. Evaporator (1→4): $q_L = h_1 - h_4$
2. Compressor (1→2): $w_{in} = h_2 - h_1$
3. Condenser (2→3): $q_H = h_2 - h_3$
4. Throttling valve (3→4): $h_4 = h_3$

\textbf{COP:}
$$COP_R = \frac{q_L}{w_{in}} = \frac{h_1 - h_4}{h_2 - h_1}$$

\textbf{Cascade Systems:}

Multiple cycles in series for large temperature differences

\textbf{Multistage Compression:}

Reduces compressor work with intercooling

\section*{Gas Mixtures}

\subsection*{Composition}

\textbf{Mass Fraction:}
$$mf_i = \frac{m_i}{m_{total}}$$

\textbf{Mole Fraction:}
$$y_i = \frac{N_i}{N_{total}}$$

\textbf{Conversion:}
$$mf_i = \frac{y_i M_i}{\sum y_j M_j}$$

\subsection*{Dalton's Law (Ideal Gases)}

$$P = \sum P_i$$

where $P_i = y_i P$ is partial pressure

\textbf{Amagat's Law:}
$$V = \sum V_i$$

where $V_i = y_i V$ is partial volume

\subsection*{Properties of Mixtures}

\textbf{Apparent Molar Mass:}
$$M_m = \sum y_i M_i$$

\textbf{Gas Constant:}
$$R_m = \frac{R_u}{M_m} = \sum mf_i R_i$$

\textbf{Specific Heat:}
$$c_{p,m} = \sum mf_i c_{p,i}$$
$$c_{v,m} = \sum mf_i c_{v,i}$$

\section*{Psychrometrics (Gas-Vapor Mixtures)}

\subsection*{Humidity Ratios}

\textbf{Absolute Humidity (Specific Humidity):}
$$\omega = \frac{m_v}{m_a} = 0.622\frac{P_v}{P - P_v}$$

where $P_v$ is partial pressure of water vapor, $P$ is total pressure

\textbf{Relative Humidity:}
$$\phi = \frac{P_v}{P_g} = \frac{actual\ vapor\ density}{saturation\ vapor\ density}$$

where $P_g$ is saturation pressure at given temperature

\textbf{Degree of Saturation:}
$$\mu = \frac{\omega}{\omega_s}$$

\subsection*{Psychrometric Properties}

\textbf{Dry-Bulb Temperature ($T_{db}$):} Actual temperature

\textbf{Wet-Bulb Temperature ($T_{wb}$):} Temperature with adiabatic saturation

\textbf{Dew-Point Temperature ($T_{dp}$):} Temperature at which condensation begins

\textbf{Enthalpy of Moist Air:}
$$h = c_p T + \omega(h_g + c_{pv}T)$$

where $h_g$ is enthalpy of vaporization at 0°C

Simplified:
$$h \approx c_p T + \omega h_g$$

\subsection*{Air-Conditioning Processes}

\textbf{Simple Heating/Cooling:}
- $\omega$ = constant
- Horizontal line on psychrometric chart

\textbf{Heating with Humidification:}
- Both $T$ and $\omega$ increase

\textbf{Cooling with Dehumidification:}
- Both $T$ and $\omega$ decrease
- Process line intersects saturation curve

\textbf{Evaporative Cooling (Adiabatic Saturation):}
- $h$ = constant (approximately)
- $\omega$ increases, $T$ decreases
- Follows constant wet-bulb temperature line

\section*{Combustion}

\subsection*{Stoichiometry}

\textbf{General Hydrocarbon Combustion:}
$$C_xH_y + a(O_2 + 3.76N_2) \rightarrow bCO_2 + cH_2O + dN_2$$

For complete combustion:
- $b = x$ (carbon balance)
- $c = y/2$ (hydrogen balance)
- $d = 3.76a$ (nitrogen balance)
- $a = x + y/4$ (oxygen balance)

\textbf{Air-Fuel Ratio (AF):}
$$AF = \frac{m_{air}}{m_{fuel}}$$

\textbf{Fuel-Air Ratio:}
$$FA = \frac{m_{fuel}}{m_{air}} = \frac{1}{AF}$$

\textbf{Percent Theoretical Air:}
$$\%TA = \frac{actual\ air}{theoretical\ air} \times 100\%$$

\textbf{Percent Excess Air:}
$$\%EA = \%TA - 100\%$$

\subsection*{Enthalpy of Formation and Combustion}

\textbf{Enthalpy of Formation ($\bar{h}_f^{\circ}$):}

Enthalpy change when compound is formed from elements at standard state (25~$^{\circ}$C, 1 atm)

For elements in standard state: $\bar{h}_f^{\circ} = 0$

\textbf{Enthalpy of Combustion ($\bar{h}_{RP}$):}

Enthalpy of reaction (negative for exothermic)

\textbf{First Law for Reacting Systems:}
$$
Q - W
=
\sum_{\text{products}} N_i(\bar{h}_f^{\circ} + \Delta\bar{h})_i
-
\sum_{\text{reactants}} N_j(\bar{h}_f^{\circ} + \Delta\bar{h})_j
$$

\subsection*{Adiabatic Flame Temperature}

For adiabatic combustion ($Q = 0$, $W = 0$):
$$
\sum_{\text{reactants}} N_j(\bar{h}_f^{\circ} + \Delta\bar{h})_j
=
\sum_{\text{products}} N_i(\bar{h}_f^{\circ} + \Delta\bar{h})_i
$$

Solve iteratively for product temperature.


\textbf{Higher Heating Value (HHV):}

Water in products is liquid

\textbf{Lower Heating Value (LHV):}

Water in products is vapor

$$HHV = LHV + m_{H_2O} \cdot h_{fg}$$

\subsection*{Entropy Change of Reacting Systems}

$$\Delta S = \sum_{products} N_i\bar{s}_i - \sum_{reactants} N_j\bar{s}_j + \Delta S_{mixing}$$

For ideal gases:
$$\bar{s}_i(T,P) = \bar{s}_i^\circ(T) - R_u\ln\left(\frac{P_i}{P_{ref}}\right)$$

\section*{Thermodynamic Relations}

\subsection*{Maxwell Relations}

From $dU = TdS - PdV$:
$$\left(\frac{\partial T}{\partial V}\right)_S = -\left(\frac{\partial P}{\partial S}\right)_V$$

From $dH = TdS + VdP$:
$$\left(\frac{\partial T}{\partial P}\right)_S = \left(\frac{\partial V}{\partial S}\right)_P$$

From $dF = -SdT - PdV$ (Helmholtz):
$$\left(\frac{\partial S}{\partial V}\right)_T = \left(\frac{\partial P}{\partial T}\right)_V$$

From $dG = -SdT + VdP$ (Gibbs):
$$\left(\frac{\partial S}{\partial P}\right)_T = -\left(\frac{\partial V}{\partial T}\right)_P$$

\subsection*{Clapeyron Equation}

$$\frac{dP_{sat}}{dT} = \frac{h_{fg}}{T v_{fg}}$$

Relates slope of saturation curve to properties

\textbf{Clausius-Clapeyron Equation (for vapor):}

$$\ln\left(\frac{P_2}{P_1}\right) = -\frac{h_{fg}}{R}\left(\frac{1}{T_2} - \frac{1}{T_1}\right)$$

\subsection*{Joule-Thomson Coefficient}

$$\mu_J = \left(\frac{\partial T}{\partial P}\right)_h$$

For ideal gas: $\mu_J = 0$

For real gas: can be positive or negative

\section*{Quick Reference}

\textbf{Common Values:}
- Standard temperature: 25°C = 298 K = 77°F
- Standard pressure: 1 atm = 101.325 kPa = 14.7 psi
- Air: $R = 0.287$ kJ/(kg·K), $c_p = 1.005$ kJ/(kg·K), $c_v = 0.718$ kJ/(kg·K), $k = 1.4$
- Water: $c_p \approx 4.18$ kJ/(kg·K)

\textbf{Sign Conventions:}
- Heat in: positive
- Work out: positive
- $Q > 0$: to system
- $W > 0$: by system

\end{document}