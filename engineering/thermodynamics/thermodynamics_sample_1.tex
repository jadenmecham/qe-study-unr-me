\documentclass[11pt]{article}
\usepackage[T1]{fontenc}
\usepackage{lmodern}
\usepackage[margin=1in]{geometry}
\usepackage{amsmath,amssymb}
\usepackage{array}
\usepackage{booktabs}
\usepackage{graphicx}

\pagestyle{empty}
\setlength{\parindent}{0pt}
\setlength{\parskip}{8pt}

\begin{document}

\begin{center}
\LARGE{\textbf{Thermodynamics Sample Problem 1}}
\end{center}

\vspace{10pt}

This is problem 1.2 on the 2023 QE.\\

In a certain industrial process, a heat engine operates between two reservoirs: a \
high-temperature reservoir at 600°C and a low-temperature reservoir at 150°C. The engine 
takes in 1500 J of heat from the high-temperature reservoir and delivers 900 J of work while 
rejecting heat to the low-temperature reservoir. Assume that the heat engine operates in a 
steady-state manner and there are no other sources of energy losses or inefficiencies in the 
engine.

\section*{Part A}

Calculate the efficiency of the heat engine.\\
The efficiency ($\eta$) of a heat engine is defined as the ratio of the work output ($W$) to
the heat input ($Q_{in}$):
\begin{equation}
    \eta = \frac{W}{Q_{in}}
\end{equation}
Substituting the given values:
\begin{equation}
    \boxed{\eta = \frac{900 \, \text{J}}{1500 \, \text{J}} = 0.6 \, \text{or} \, 60\%}
\end{equation}

\section*{Part B}
Determine the amount of heat rejected to the low-temperature reservoir.\\
The heat rejected ($Q_{out}$) to the low-temperature reservoir can be calculated using the
first law of thermodynamics for a heat engine:
\begin{equation}
    Q_{in} = W + Q_{out}
\end{equation}
Rearranging the equation to solve for $Q_{out}$:
\begin{equation}
    Q_{out} = Q_{in} - W
\end{equation}
Substituting the given values:
\begin{equation}
    \boxed{Q_{out} = 1500 \, \text{J} - 900 \, \text{J} = 600 \, \text{J}}
\end{equation}

\section*{Part C}
Calculate the change in entropy of the high-temperature reservoir, 
low-temperature reservoir, and the overall entropy change for the entire process.

The change in entropy ($\Delta S$) for a reservoir can be calculated using the formula:
\begin{equation}
    \Delta S = \frac{Q}{T}
\end{equation}
Where $Q$ is the heat exchanged and $T$ is the absolute temperature in Kelvin.
First, we need to convert the temperatures from Celsius to Kelvin:
\begin{gather}
    T_{high} = 600 + 273.15 = 873.15 \, \text{K} \\
    T_{low} = 150 + 273.15 = 423.15 \, \text{K}
\end{gather}

Now, we can calculate the change in entropy for each reservoir:
\begin{gather}
    \Delta S_{high} = \frac{-Q_{in}}{T_{high}} = \frac{-1500 \, \text{J}}{873.15 \, \text{K}} \approx -1.72 \, \text{J/K} \\
    \Delta S_{low} = \frac{Q_{out}}{T_{low}} = \frac{600 \, \text{J}}{423.15 \, \text{K}} \approx 1.42 \, \text{J/K}
\end{gather}
The overall change in entropy for the entire process is the sum of the changes in entropy of
both reservoirs:
\begin{equation}
    \Delta S_{total} = \Delta S_{high} + \Delta S_{low} \approx -1.72 \, \text{J/K} + 1.42 \, \text{J/K} \approx -0.30 \, \text{J/K}
\end{equation}
\begin{equation}
    \boxed{\Delta S_{total} \approx -0.30 \, \text{J/K}}
\end{equation}

\section*{Part D}
Discuss the feasibility of achieving a higher efficiency for the given temperature reservoirs.\\

To evaluate the feasibility of achieving a higher efficiency, we can compare the calculated
efficiency of the heat engine to the Carnot efficiency, which represents the maximum possible
efficiency for a heat engine operating between two temperature reservoirs. The Carnot efficiency
($\eta_{Carnot}$) is given by:
\begin{equation}
    \eta_{Carnot} = 1 - \frac{T_{low}}{T_{high}}
\end{equation}
Substituting the temperatures in Kelvin:
\begin{equation}
    \eta_{Carnot} = 1 - \frac{423.15 \, \text{K}}{873.15 \, \text{K}} \approx 0.515 \, \text{or} \, 51.5\%
\end{equation}
Since the calculated efficiency of the heat engine (60\%) exceeds the Carnot efficiency
(51.5\%), it indicates that the engine is operating beyond the theoretical maximum efficiency,
which is not feasible according to the second law of thermodynamics. Therefore, achieving a
higher efficiency than the Carnot efficiency is impossible for the given temperature reservoirs.

\end{document}