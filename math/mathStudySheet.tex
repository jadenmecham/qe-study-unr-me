\documentclass[12pt]{article}

% Packages
\usepackage{amsmath, amssymb}
\usepackage{geometry}
\usepackage[most]{tcolorbox}
\geometry{margin=1in}

% Header info
\title{Study Sheet}
\author{Based off example problems from 21 and 23}
% \date{}

\begin{document}
\maketitle
%%%%%%%%%%%%%%%%%%%%%%%%%%%%%%%%%%%%%%%%%%%%%%%%%%%%%%%%%%%%%%%%%%%%%%%%%%%%%%%%%%%
\newpage
\section{Vector Calculus}
\textbf{Observed Methods}
\begin{itemize}
    \item Work along a path
    \item Stokes theorem
    \item Divergence theorem
    \item Cylindrical bounds of integration
    \item Volume of a sphere
\end{itemize}

\noindent\hrulefill

\begin{tcolorbox}[colback=blue!3!white, colframe=black!10!white, title=\textcolor{black}{Line/Work Integral},breakable]
Generally given a force field $F(x,y)$ and a curve through space $\mathcal{C}$, for which we want to compute 
\begin{align*}
    \int_\mathcal{C}\vec F\cdot d\vec r
\end{align*}
Which is the \textbf{work} of the vector field along the path. To simplify the integration, 
\begin{align*}
    d\vec r \frac{dt}{dt} \to \frac{d\vec r}{dt}dt \to \vec v dt
\end{align*}
Before integrating, both $F$ and $\mathcal{C}$ need to be \textit{parameterized} in terms of $t$. Any definition for $x$ works, but the algebra can be difficult. So the simplest method may be
\begin{align*}
    \mathcal{C}(x,y) = r(t)
    \begin{cases}
        x\equiv t\\
        y\equiv \mathcal{C}(x=t)
    \end{cases}
\end{align*}
Plug these values into the vector field $F$. 
\begin{align*}
    F(x,y) = F(t)
    \begin{cases}
        x\equiv t\\
        y\equiv \mathcal{C}(x=t)
    \end{cases}
\end{align*}
Then the last step before integration is to take the derivative of $r(t)$ to get $v(t)$. Finally, the \textbf{line/work} integral can be computed via
\begin{align*}
    \boxed{\int_\mathcal{C}\vec F(x,y)\cdot d\vec r(x,y) = \int_a^b F(t) v(t)dr(t)}
\end{align*}
\end{tcolorbox}

\begin{tcolorbox}[colback=blue!3!white, colframe=blue!10!white, title=\textcolor{black}{Divergence Theorem},breakable]
The divergence of a vector field, $\nabla \cdot \vec F$, integrated over a volume, $V$ is equal to the dot product between the vector field and the surface of the volume, $\vec F \cdot \vec n$ ($\vec n$ is the surface normal vector), integrated over the surface area of the volume, $A$.
\begin{align*}
    \boxed{\int_V \nabla \cdot \vec F dV = \int_S \vec F \cdot \vec n dA}
\end{align*}
Solving the \textbf{RHS} amounts to finding the surface-normal vector $\vec n$. The general method is to fist parameterize the given geometries defining the boundary of the volume.
\begin{align*}
    S =&\; f(x,y,z) \to z=f(x,y) \\
    S =&\; \left[f(u),f(v),f(u,v)\right] 
    \begin{cases}
        x \equiv f(u)\\
        y \equiv f(v)
    \end{cases}
\end{align*}
With the surface parameterized, take the cross product between the two respective partial derivatives
\begin{align*}
    \begin{vmatrix}
        \hat i & \hat j & \hat k \\
        \frac{\partial S}{\partial u}|_{\hat i} & \frac{\partial S}{\partial u}|_{\hat j} & \frac{\partial S}{\partial u}|_{\hat k} \\
        \frac{\partial S}{\partial v}|_{\hat i} & \frac{\partial S}{\partial v}|_{\hat j} & \frac{\partial S}{\partial v}|_{\hat k}
    \end{vmatrix}
    = \vec n
\end{align*}
Take note that the direction of $\vec n$ needs to \textit{face out of the surface}, and the cross product may have it facing the wrong direction. 

\end{tcolorbox}


%%%%%%%%%%%%%%%%%%%%%%%%%%%%%%%%%%%%%%%%%%%%%%%%%%%%%%%%%%%%%%%%%%%%%%%%%%%%%%%%%%%
\newpage
\section{Linear Algebra}
\textbf{Observed Methods}
\begin{itemize}
    \item Solutions to systems of ODE
    \item Matrix power
    \item Inverse
    \item adjudicate 
    \item Co-factor
    \item Eig-stuff
\end{itemize}

\noindent\hrulefill


%%%%%%%%%%%%%%%%%%%%%%%%%%%%%%%%%%%%%%%%%%%%%%%%%%%%%%%%%%%%%%%%%%%%%%%%%%%%%%%%%%%
\newpage
\section{Laplace Transforms}
\textbf{Observed Methods}
\begin{itemize}
    \item Convolution transform 
    \item Derivative transform
    \item Partial fractions
    \item Integration by parts
\end{itemize}


\noindent\hrulefill

\begin{tcolorbox}[colback=blue!3!white, colframe=blue!10!white, title=\textcolor{black}{Convolution Theorem},breakable]
The convolution of two functions is given by the integral
\begin{align*}
    (f \circ g)(t) = h(t) = \int_0^t f(\tau) g(t-\tau)d\tau
\end{align*}
The Laplace transform of a convolution is
\begin{align*}
    \boxed{\mathcal{L}[h(t)] = \mathcal{L}[f(t)]\mathcal{L}[g(t)] = FG}
\end{align*}
The inverse of this theorem can be used if a function in Laplace space, $H$, can be observed as a product of two functions, $F$ and $G$, for which the transforms are known. The inverse would be
\begin{align*}
    H(s) =&\; F(s)G(s)\\
    \mathcal{L}[H]=&\;\mathcal{L}[FG] = \int_0^t f(\tau) g(t-\tau)d\tau = h(t)
\end{align*}
\end{tcolorbox}

\begin{tcolorbox}[colback=blue!3!white, colframe=blue!10!white, title=\textcolor{black}{Derivatives of Laplace},breakable]
Consider the $n^{th}$ order differential 
\begin{align*}
    x^n = \frac{d^n x}{dt^n}
\end{align*}
Taking the Laplace transform would give
\begin{align*}
    \boxed{\mathcal{L}[x^n] =  s^nX -s^{n-1}x_0 - s^{n-1}\dot x_0 -...- sx^{n-2}_0-x^{n-1}_0}
\end{align*}
where $X$ is the Laplace transform of $x$, and each $x_0^j$ is the initial condition for each derivative of $x$. The third order example would be 
\begin{align*}
    \mathcal{L}\left[ \dddot x\right]= s^3X - s^2x_0 - s\dot x_0 -\ddot x_0
\end{align*}
\end{tcolorbox}

%%%%%%%%%%%%%%%%%%%%%%%%%%%%%%%%%%%%%%%%%%%%%%%%%%%%%%%%%%%%%%%%%%%%%%%%%%%%%%%%%%%
\newpage
\section{Fourier Transforms}
\textbf{Observed Methods}
\begin{itemize}
    \item Cosine Series
    \item Complex Transform
    \item Integration by parts
    \item General Series
    \item odd / even function cancellations
\end{itemize}
\noindent\hrulefill
\begin{tcolorbox}[colback=blue!3!white, colframe=red!10!white, title=\textcolor{black}{General Series Expansion},breakable]
A function that is piecewise smooth, and has a period length $p=2L$, can be approximated by a series of sines and cosines
\begin{align*}
    f(x) = &\; a_0 + \sum_{n=1}^\infty \left(a_n \cos \frac{n\pi x}{L} + b_n \sin \frac{n\pi x}{L} \right)\\
    &\; a_0 = \frac{1}{2L}\int_{-L}^L f(x)dx\\
    &\; a_n = \frac{1}{L}\int_{-L}^L f(x)\cos \frac{n\pi x}{L} \hspace{0.1cm}dx\\
    &\; b_n = \frac{1}{L}\int_{-L}^L f(x)\sin \frac{n\pi x}{L} \hspace{0.1cm}dx\\
\end{align*}
\end{tcolorbox}

\begin{tcolorbox}[colback=blue!3!white, colframe=black!10!white, title=\textcolor{black}{Orthogonality of sin and cos},breakable]
The product of trig functions are orthogonal on the interval $-\pi \leq x \leq \pi$, specifically:
\begin{align*}
    \begin{cases}
        \int_{-\pi}^\pi \cos nx \cos mx \hspace{0.1cm} dx & (n\neq m)\\
        \int_{-\pi}^\pi \sin nx \sin mx \hspace{0.1cm} dx & (n\neq m)\\
        \int_{-\pi}^\pi \sin nx \cos mx \hspace{0.1cm} dx & (\forall n,m)\\
    \end{cases}
\end{align*}
Use this to reduce the number of coefficients that you need to compute in the Fourier Series, when $f(x)$ has trig terms 
\end{tcolorbox}

\begin{tcolorbox}[colback=blue!3!white, colframe=red!10!white, title=\textcolor{black}{Transforms},breakable]
\textbf{Complex Transform}
\begin{align*}
    \hat f(w) =&\; \frac{1}{\sqrt{2\pi}}\int_{-\infty}^\infty f(x)e^{-iwx}dx\\
    f(x) =&\; \frac{1}{\sqrt{2\pi}}\int_{-\infty}^\infty \hat f(w)e^{iwx}dx
\end{align*}
\textbf{Sine Transform}
\begin{align*}
    \hat f_s(w) =&\; \sqrt{\frac{2}{\pi}}\int_{0}^\infty f(x)\sin wx\hspace{0.1cm}dx\\
    f(x) =&\; \sqrt{\frac{2}{\pi}}\int_{0}^\infty \hat f(w)\sin wx\hspace{0.1cm}dw\\
\end{align*}
\textbf{Cosine Transform}
\begin{align*}
    \hat f_c(w) =&\; \sqrt{\frac{2}{\pi}}\int_{0}^\infty f(x)\cos wx\hspace{0.1cm}dx\\
    f(x) =&\; \sqrt{\frac{2}{\pi}}\int_{0}^\infty \hat f(w)\cos wx\hspace{0.1cm}dw\\
\end{align*}
\end{tcolorbox}



%%%%%%%%%%%%%%%%%%%%%%%%%%%%%%%%%%%%%%%%%%%%%%%%%%%%%%%%%%%%%%%%%%%%%%%%%%%%%%%%%%%
\newpage
\section{Ordinary Diff. Eq.s}
\textbf{Observed Methods}
\begin{itemize}
    \item Undetermined coefficients
    \item Cauchy-Euler Eq
    \item Log properties
\end{itemize}


\noindent\hrulefill

%%%%%%%%%%%%%%%%%%%%%%%%%%%%%%%%%%%%%%%%%%%%%%%%%%%%%%%%%%%%%%%%%%%%%%%%%%%%%%%%%%%
\newpage
\section{Partial Diff. Eq.s}
\textbf{Observed Methods}
\begin{itemize}
    \item Infinite Rode (Heat Eq.)
    \item String (Wave Eq.)
    \item Separation of Variables
\end{itemize}


\noindent\hrulefill


%%%%%%%%%%%%%%%%%%%%%%%%%%%%%%%%%%%%%%%%%%%%%%%%%%%%%%%%%%%%%%%%%%%%%%%%%%%%%%%%%%%
\newpage
\section{General Math Techniques}
\textbf{Observed Methods}
\begin{itemize}
    \item Cross product
    \item Integration by Parts
    \item Partial Fractions
    \item Completing the square
    \item Quadratic equation 
\end{itemize}


\noindent\hrulefill


\end{document}

% \noindent\hrulefill

\begin{tcolorbox}[colback=blue!3!white, colframe=red!10!white, title=\textcolor{black}{Important Equation},breakable]

\end{tcolorbox}

\begin{tcolorbox}[colback=blue!3!white, colframe=black!10!white, title=\textcolor{black}{Math tep/tricks},breakable]

\end{tcolorbox}

\begin{tcolorbox}[colback=blue!3!white, colframe=blue!10!white, title=\textcolor{black}{Theorem/Concept},breakable]

\end{tcolorbox}