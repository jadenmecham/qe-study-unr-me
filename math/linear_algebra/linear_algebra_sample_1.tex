\documentclass[11pt]{article}
\usepackage[T1]{fontenc}
\usepackage{lmodern}
\usepackage[margin=1in]{geometry}
\usepackage{amsmath,amssymb}
\usepackage{array}
\usepackage{booktabs}

\pagestyle{empty}
\setlength{\parindent}{0pt}
\setlength{\parskip}{8pt}

\begin{document}

\begin{center}
\LARGE{\textbf{Linear Algebra Sample Problem 1}}
\end{center}

\vspace{10pt}

This problem is question 4.1 on the Fall 2023 QE.\\

Consider the matrix $A = \begin{bmatrix} 1&1&0\\0&1&0\\0&0&2 \end{bmatrix}$. Compute
$e^{At}$.

\section{Solution Method}

Here, we will use the laplace transform method to compute the matrix exponential. The laplace
transform method works well here because it works regardless of whether the matrix is
diagonalizable or not. To find the matrix exponential, this method states:
\begin{equation}
    e^{At} = \mathcal{L}^{-1} \left\{ (sI - A)^{-1} \right\}
\end{equation}

Also note the that the inverse of a matrix can be computed using the formula:
\begin{equation}
    M^{-1} = \frac{\text{adj}(M)}{\det(M)}
\end{equation}

\section(Compute $(sI-A)$)
First, we compute $sI - A$:
\begin{equation}
    sI - A = \begin{bmatrix}
        s-1 & -1 & 0 \\
        0 & s-1 & 0 \\
        0 & 0 & s-2
    \end{bmatrix}
\end{equation}

\section{Compute $\det(sI - A)$}
Next, we compute the determinant of $sI - A$:
\begin{equation}
    \det(sI - A) = (s-1)^2 (s-2)
\end{equation}

\section{Compute $\text{adj}(sI - A)$}

Now, we need to find the cofactor matrix of $sI - A$. Each element of the cofactor matrix
is computed using the formula:
\begin{equation}
    C_{ij} = (-1)^{i+j} \text{det}(M_{ij})
\end{equation}

The final cofactor matrix is:
\begin{equation}
    C = \begin{bmatrix}
        (s-1)(s-2) & (s-1)(0) & 0 \\
        s-2& (s-1)(s-2) & 0 \\
        0 & 0 & (s-1)^2
    \end{bmatrix} = \begin{bmatrix}
        (s-1)(s-2) & 0 & 0 \\
        s-2 & (s-1)(s-2) & 0 \\
        0 & 0 & (s-1)^2
    \end{bmatrix}
\end{equation}

Taking the transpose of the cofactor matrix gives us the adjugate matrix:
\begin{equation}
    \text{adj}(sI - A) = C^T = \begin{bmatrix}
        (s-1)(s-2) & s-2 & 0 \\
        0 & (s-1)(s-2) & 0 \\
        0 & 0 & (s-1)^2
    \end{bmatrix}
\end{equation}

\section{Compute $(sI - A)^{-1}$}
Now, we can compute the inverse of $sI - A$:
\begin{equation}
    (sI - A)^{-1} = \frac{\text{adj}(sI - A)}{\det(sI - A)} = \begin{bmatrix}
        \frac{1}{s-1} & \frac{1}{(s-1)^2} & 0 \\
        0 & \frac{1}{s-1} & 0 \\
        0 & 0 & \frac{1}{s-2}
    \end{bmatrix}
\end{equation}

\section{Compute $e^{At}$}
Finally, we can compute the matrix exponential by taking the inverse laplace transform of
$(sI - A)^{-1}$. This is just the inverse laplace transform of each element of the matrix:
\begin{equation}
    e^{At} = \mathcal{L}^{-1} \left\{ (sI - A)^{-1} \right\} = \begin{bmatrix}
        e^{t} & t e^{t} & 0 \\
        0 & e^{t} & 0 \\
        0 & 0 & e^{2t}
    \end{bmatrix}
\end{equation}

\end{document}