\documentclass[11pt]{article}
\usepackage[T1]{fontenc}
\usepackage{lmodern}
\usepackage[margin=1in]{geometry}
\usepackage{amsmath,amssymb}
\usepackage{array}
\usepackage{booktabs}

\pagestyle{empty}
\setlength{\parindent}{0pt}
\setlength{\parskip}{8pt}

\begin{document}

\begin{center}
\LARGE{\textbf{Linear Algebra Sample Problem 2}}
\end{center}

\vspace{10pt}

This problem is question 4.2 on the Fall 2023 QE.\\

Given the matrix $ A = \begin{bmatrix} 0&1\\1+a&b\end{bmatrix}$ calculate the eigenvalues
and eigenvectors. Find the values for $a$ and $b$ for which the eigenvalues are real 
part negative. 

\section{Find the Eigenvalues}

To find the eigenvalues of the matrix $A$, we need to solve the characteristic equation given by:
\begin{equation}
    \det(A - \lambda I) = 0
\end{equation}

First, we compute $A - \lambda I$:
\begin{equation}
    A - \lambda I = \begin{bmatrix}
        -\lambda & 1 \\
        1 + a & b - \lambda
    \end{bmatrix}
\end{equation}

Next, we compute the determinant:
\begin{equation}
    \det(A - \lambda I) = (-\lambda)(b - \lambda) - (1)(1 + a) = \lambda^2 - b\lambda - (1 + a)
\end{equation}

Setting the determinant equal to zero gives us the characteristic equation:
\begin{equation}
    \lambda^2 - b\lambda - (1 + a) = 0
\end{equation}

To find the eigenvalues, we can use the quadratic formula.
\begin{equation}
    \lambda_{1,2} = \frac{b \pm \sqrt{b^2 + 4(1 + a)}}{2}
\end{equation}

\section{Find the Eigenvectors}

To find the eigenvectors corresponding to each eigenvalue, we solve the equation:
\begin{equation}
    (A - \lambda I)\mathbf{v} = 0
\end{equation}

For the given matrix, this looks like:
\begin{equation}
    \begin{bmatrix}
        -\lambda & 1 \\
        1 + a & b - \lambda
    \end{bmatrix}
    \begin{bmatrix}
        v_1 \\
        v_2
    \end{bmatrix} = 0
\end{equation}

This leads to the system of equations:
\begin{gather}
    -\lambda v_1 + v_2 = 0 \\
    (1 + a)v_1 + (b - \lambda)v_2 = 0
\end{gather}

From the first equation, we can express $v_2$ in terms of $v_1$:
\begin{equation}
    v_2 = \lambda v_1
\end{equation}

We can then rewrite the $v$ vector as:
\begin{equation}
v = 
    \begin{bmatrix}
        1\\ \lambda
    \end{bmatrix}
\end{equation}

We can then sibstitute in the expression for $\lambda$ to get the eigenvectors.
\begin{gather}
v_{1,2} = 
    \begin{bmatrix}
        1\\ \frac{b \pm \sqrt{b^2 + 4(1 + a)}}{2}
    \end{bmatrix}\\
=    \begin{bmatrix}
        2\\ b \pm \sqrt{b^2 + 4(1 + a)}
    \end{bmatrix}
\end{gather}

\section{Conditions for Negative Real Parts of Eigenvalues}

For the eigenvalues to have negative real parts, we need to analyze the expression:
\begin{equation}
    \lambda_{1,2} = \frac{b \pm \sqrt{b^2 + 4(1 + a)}}{2}
\end{equation}

To ensure that both eigenvalues have negative real parts, we need:
\begin{enumerate}
    \item The real part of both eigenvalues must be negative, which implies:
    \begin{equation}
        b < 0
    \end{equation}
    \item The discriminant must be non-positive to ensure complex eigenvalues with negative real parts:
    \begin{equation}
        b^2 + 4(1 + a) \leq 0 \implies a \leq -1 - \frac{b^2}{4}
    \end{equation}
\end{enumerate}

\end{document}