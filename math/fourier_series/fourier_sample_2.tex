\documentclass[11pt]{article}
\usepackage[T1]{fontenc}
\usepackage{lmodern}
\usepackage[margin=1in]{geometry}
\usepackage{amsmath,amssymb}
\usepackage{array}
\usepackage{booktabs}

\pagestyle{empty}
\setlength{\parindent}{0pt}
\setlength{\parskip}{8pt}

\begin{document}

\begin{center}
\LARGE{\textbf{Fourier Series and Transforms Sample Problem 2}}
\end{center}

\vspace{10pt}

This problem is question 3.2 on the Fall 2023 QE.\\

The rectangular function is defined as:
\begin{equation}
    \text{rect}(t/a) = \begin{cases}
        0, & |t| > \frac{a}{2} \\
        \frac{1}{2}, & |t| = \frac{a}{2}\\
        1, & |t| < \frac{a}{2}
    \end{cases}
\end{equation}

With $a > 0$. Compute the Fourier transform of the signal and sketch the 
magnitude of the spectrum signal, as a function of the parameter $a$.

\section{Fourier Transform Definition}
The Fourier transform $X(f)$ of a time-domain signal $x(t)$ is defined as:
\begin{equation}
    X(f) = \int_{-\infty}^{\infty} x(t) e^{-j \omega t} \, dt
\end{equation}
Where $\omega = 2 \pi f$ is the angular frequency.

\section{Compute the Fourier Transform of $\text{rect}(t)$}
To compute the Fourier transform of the rectangular function, 
we substitute $x(t) = \text{rect}(t)$ into the Fourier transform definition:
\begin{equation}
    X(f) = \int_{-\infty}^{\infty} \text{rect}(t/a) e^{-j \omega t} \, dt
\end{equation}
Since $\text{rect}(t)$ is non-zero only in the interval $[-\frac{a}{2}, \frac{a}{2}]$,
we can limit the integration bounds. Furthermore, within this interval
, $\text{rect}(t/a) = 1$. Thus, we have:
\begin{equation}
    X(f) = \int_{-\frac{a}{2}}^{\frac{a}{2}} e^{-j \omega t} \, dt
\end{equation}

Evaluating this integral, we get:
\begin{equation}
    X(f) = \left[ \frac{e^{-j \omega t}}{-j \omega} \right]_{-\frac{a}{2}}^{\frac{a}{2}} 
    = \frac{1}{-j \omega} \left( e^{-j \omega \frac{a}{2}} - e^{j \omega \frac{a}{2}} \right)
\end{equation}

Simplifying the expression using Euler's formula, we find:
\begin{equation}
    X(f) = \frac{2 \sin\left( \frac{\omega a}{2} \right)}{\omega} 
    = \frac{2 \sin\left( \pi f a \right)}{2 \pi f} 
    = a \, \text{sinc}(f a)
\end{equation}

For context, Euler's formula is:
\begin{equation}
    e^{j \theta} = \cos(\theta) + j \sin(\theta)
\end{equation}

\end{document}