\documentclass[11pt]{article}
\usepackage[T1]{fontenc}
\usepackage{lmodern}
\usepackage[margin=1in]{geometry}
\usepackage{amsmath,amssymb}
\usepackage{array}
\usepackage{booktabs}

\pagestyle{empty}
\setlength{\parindent}{0pt}
\setlength{\parskip}{8pt}

\begin{document}

\begin{center}
\LARGE{\textbf{Partial Differential Equations Sample Problem 1}}
\end{center}

\vspace{10pt}

This problem is question 6.1 on the 2023 QE. 

Transverse vibrations of an idealized guitar string are described by the equation:
\begin{equation}
    \rho\frac{\partial^2u(x,t)}{\partial t^2} - T\frac{\partial^2u(x,t)}{\partial x^2} = 0
\end{equation}

with $u$ the trasverse displacement, $\rho$ the constant string density per unit length, 
and $T$ the constant string tension. Assume fixed boundary condition given by 
$u(0,t)=u(L,t)=0$. At $t=0$, the string is plucvked at a point $x=2L/3$ and released
from rest. For simpicity, assume that the initial deformed shape is triangular and 
descirbed by $u(x,0)=3Ax/(2L)$ for $0<x<2L/3$ and $u(x,0)=3A(L-x)/L$ for $L/3<x<L$. 
Determine the ensuing string vibrations using separation of variables.

\section{Rewrite in Standard Form}

First, we rewrite the equation in standard form to isolate the second time derivative.

\begin{equation}
    \frac{\partial^2u}{\partial t^2} = \frac{T}{\rho}\cdot \frac{\partial^2u}{\partial x^2}
\end{equation}

For simplicity, we will let $c^2 = \frac{T}{\rho}$, so that the equation becomes:
\begin{equation}
    \frac{\partial^2u}{\partial t^2} = c^2 \frac{\partial^2u}{\partial x^2}
\end{equation}

\section{Separation of Variables}
First, we will assume that the solution has the form:
\begin{equation}
    u(x,t) = X(x)T(t)
\end{equation}

We can then substitute this into our standard form equation.
\begin{equation}
    X(x)T''(t) = c^2 X''(x)T(t)
\end{equation}

Then, rearrange the equation to isolate the variables again and set them equal to
a separation constant $-\lambda$.
\begin{equation}
    \frac{T''(t)}{ T(t)} = c^2\frac{X''(x)}{X(x)} = -\lambda
\end{equation}

\section{Solve the Spatial Equation}

We will start by solving the $X$ equation. First, rearrange it to make it equal to zero.
\begin{equation}
    X'' + \frac{\lambda}{c^2}X = 0
\end{equation}

We will let $k^2 = \frac{\lambda}{c^2}$, so that the equation becomes:
\begin{equation}
    X'' + k^2 X = 0
\end{equation}

The general solution to this equation is:
\begin{equation}
    X(x) = A\cos(kx) + B\sin(kx)
\end{equation}

We can then apply the boundary conditions to find the value of $A$.
Applying the boundary condition at $x=0$:
\begin{equation}
    X(0) = A\cos(0) + B\sin(0) = A = 0
\end{equation}
Thus, the solution simplifies to:
\begin{equation}
    X(x) = B\sin(kx)
\end{equation}

Next, we apply the boundary condition at $x=L$:
\begin{equation}
    X(L) = B\sin(kL) = 0
\end{equation}
For a non-trivial solution, we require that $\sin(kL) = 0$, which implies:
\begin{equation}
    kL = n\pi \quad \text{for } n=1,2,3,\ldots
\end{equation}
Thus, we have:
\begin{equation}
    k = \frac{n\pi}{L}
\end{equation}

We can substitute back into our spatial solution to get:
\begin{equation}
    X_n(x) = B\sin\left(\frac{n\pi x}{L}\right)
\end{equation}

\section{Solve the Temporal Equation}

Next, we will solve the $T$ equation. First, rearrange it to make it equal to zero.
\begin{equation}
    T'' + \lambda T = 0
\end{equation}

The general solution to this equation is:
\begin{equation}
    T_n(t) = C\cos\left(\frac{n\pi c t}{L}\right) + D\sin\left(\frac{n\pi c t}{L}\right)
\end{equation}

\section{Combine Solutions}
We can now combine the spatial and temporal solutions to get the general solution for $u(x,t)$:
\begin{equation}
    u(x,t) = \sum_{n=1}^{\infty} \left[ B_n \sin\left(\frac{n\pi x}{L}\right) \left( C_n \cos\left(\frac{n\pi c t}{L}\right) + D_n \sin\left(\frac{n\pi c t}{L}\right) \right) \right]
\end{equation}
\section{Apply Initial Conditions}
We will now apply the initial conditions to determine the coefficients $B_n$, $C_n$, and $D_n$.
The initial displacement condition is given by:
\begin{equation}
    u(x,0) = \sum_{n=1}^{\infty} B_n C_n \sin\left(\frac{n\pi x}{L}\right) = 
    \begin{cases} 
      \frac{3Ax}{2L} & 0 < x < \frac{2L}{3} \\ 
      \frac{3A(L-x)}{2L} & \frac{2L}{3} < x < L 
    \end{cases}
\end{equation}
To find the coefficients $B_n C_n$, we can use the orthogonality of the sine functions:
\begin{equation}
    B_n C_n = \frac{2}{L} \int_0^L u(x,0) \sin\left(\frac{n\pi x}{L}\right) dx
\end{equation}
Calculating this integral piecewise, we have:
\begin{equation}
    B_n C_n = \frac{2}{L} \left( \int_0^{\frac{2L}{3}} \frac{3Ax}{2L} \sin\left(\frac{n\pi x}{L}\right) dx + \int_{\frac{2L}{3}}^L \frac{3A(L-x)}{2L} \sin\left(\frac{n\pi x}{L}\right) dx \right)
\end{equation}
Evaluating these integrals will yield the coefficients $B_n C_n$.
The initial velocity condition is given by:
\begin{equation}
    \frac{\partial u}{\partial t}(x,0) = \sum_{n=1}^{\infty} B_n D_n \frac{n\pi c}{L} \sin\left(\frac{n\pi x}{L}\right) = 0
\end{equation}
Since the initial velocity is zero, we have:
\begin{equation}
    B_n D_n = 0 \quad \text{for all } n
\end{equation}
Thus, we conclude that $D_n = 0$ for all $n$.
\section{Final Solution}
The final solution for the string vibrations is given by:
\begin{equation}
    u(x,t) = \sum_{n=1}^{\infty} B_n C_n \sin\left(\frac{n\pi x}{L}\right) \cos\left(\frac{n\pi c t}{L}\right)
\end{equation}

\end{document}