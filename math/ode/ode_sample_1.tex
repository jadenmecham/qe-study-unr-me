\documentclass[11pt]{article}
\usepackage[T1]{fontenc}
\usepackage{lmodern}
\usepackage[margin=1in]{geometry}
\usepackage{amsmath,amssymb}
\usepackage{array}
\usepackage{booktabs}

\pagestyle{empty}
\setlength{\parindent}{0pt}
\setlength{\parskip}{8pt}

\begin{document}

\begin{center}
\LARGE{\textbf{ODEs Sample Problem 1}}
\end{center}

\vspace{10pt}

This problem is question 2.1 from the Fall 2023 QE exam:\\

Solve the follwoing differential equation:
\begin{gather}
    x^2y''+axy'+by=0\\
    a=-1.5, \quad b=-1.5
\end{gather}

\section{Cauchy-Euler Equation}
This is a Cauchy-Euler equation, where we can assume the solution is of the form:
\begin{equation}
    y=x^r
\end{equation}

We can then find the ffirst two derivatives.
\begin{gather}
    y'=rx^{r-1}\\
    y''=r(r-1)x^{r-2}
\end{gather}

Then, substitute these into the original equation. 

\begin{gather}
    x^2(r(r-1)x^{r-2}) + ax(rx^{r-1}) + b(x^r) = 0\\
    r(r-1)x^r + arx^r + bx^r = 0\\
    x^r[r(r-1) + ar + b] = 0
\end{gather}

\section{Characteristic Equation}
Since \( x^r \neq 0 \), we can set the term in brackets to zero:
\begin{gather}
    r(r-1) + ar + b = 0\\
    r^2 - r + ar + b = 0\\
    r^2 + (a-1)r + b = 0
\end{gather}
This is the characteristic equation. Substituting in the values for \( a \) and \( b \):
\begin{equation}
    r^2 + (-1.5-1)r - 1.5 = 0
\end{equation}

We can multiply through by 2 to eliminate the fraction:
\begin{equation}
    2r^2 - 5r - 3 = 0
\end{equation}

Using the quadratic formula to solve for \( r \):
\begin{gather}
    r = \frac{-(-5) \pm \sqrt{(-5)^2 - 4(2)(-3)}}{2(2)}\\
    r = \frac{5 \pm \sqrt{25 + 24}}{4}\\
    r = \frac{5 \pm 7}{4}
\end{gather}
This gives us two roots:
\begin{gather}
    r_1 = \frac{12}{4} = 3\\
    r_2 = \frac{-2}{4} = -0.5
\end{gather}

\section{solution}
Since we have two distinct real roots, the general solution is given by:
\begin{equation}
    y = C_1 x^{r_1} + C_2 x^{r_2}
\end{equation}
Substituting in the values for \( r_1 \) and \( r_2 \):
\begin{equation}
    y = C_1 x^3 + C_2 x^{-0.5}
\end{equation}

\end{document}