\documentclass[11pt]{article}
\usepackage[T1]{fontenc}
\usepackage{lmodern}
\usepackage[margin=1in]{geometry}
\usepackage{amsmath,amssymb}
\usepackage{array}
\usepackage{booktabs}

\pagestyle{empty}
\setlength{\parindent}{0pt}
\setlength{\parskip}{8pt}

\begin{document}

\begin{center}
\LARGE{\textbf{ODEs Sample Problem 2}}
\end{center}

\vspace{10pt}

This problem is question 2.2 on the 2023 Fall QE. \\

Solve the following second order differential equation:
\begin{gather}
    \ddot{x}=x(t)+e^{-t}+\cos(t)\\
    x(0)=1, \quad \dot{x}(0)=0
\end{gather}

Important note: Do not use Laplace methods. 

\section{Homogeneous Solution}

We can rewrite the equation as the following to isolate x variables on one side:
\begin{gather}
    \ddot{x}-x(t)=e^{-t}+\cos(t)\\
    x(0)=1, \quad \dot{x}(0)=0
\end{gather}

Now, write the characteristic equation with the form:
\begin{gather}
    ar^2+br+c=0
\end{gather}

For our case, we have:
\begin{gather}
    r^2-1=0
\end{gather}

Solving for r, we find:
\begin{gather}
    r=\pm 1
\end{gather}

Thus, the homogeneous solution is:
\begin{gather}
    x_h(t)=C_1 e^{t}+C_2 e^{-t}
\end{gather}

\section{Particular Solution for $x''-x=e^{-t}$}

To find a particular solution for the non-homogeneous part $e^{-t}$, we can use the method of 
undetermined coefficients. Since $e^{-t}$ is already part of the homogeneous solution, 
we multiply by t to find a suitable form for the particular solution:
\begin{gather}
    x_{p1}(t)=At e^{-t}
\end{gather}
Taking the first and second derivatives, we have:
\begin{gather}
    \dot{x}_{p1}(t)=Ae^{-t}-At e^{-t}\\
    \ddot{x}_{p1}(t)=-2Ae^{-t}+At e^{-t}
\end{gather}

Substituting into the differential equation:
\begin{gather}
    \ddot{x}_{p1}-x_{p1}=-2Ae^{-t}+At e^{-t}-At e^{-t}=-2Ae^{-t}
\end{gather}

Setting this equal to $e^{-t}$, we find:
\begin{gather}
    -2A=1 \implies A=-\frac{1}{2}
\end{gather}

Thus, the particular solution for this part is:
\begin{gather}
    x_{p1}(t)=-\frac{1}{2}t e^{-t}
\end{gather}

\section{Particular Solution for $x''-x=\cos(t)$}
Next, we find a particular solution for the non-homogeneous part $\cos(t)$.
We can use the method of undetermined coefficients again, proposing a solution of the form:
\begin{gather}
    x_{p2}(t)=B\cos(t)+C\sin(t)
\end{gather}

Taking the first and second derivatives, we have:
\begin{gather}
    \dot{x}_{p2}(t)=-B\sin(t)+C\cos(t)\\
    \ddot{x}_{p2}(t)=-B\cos(t)-C\sin(t)
\end{gather}

Substituting into the differential equation:
\begin{gather}
    \ddot{x}_{p2}-x_{p2}=-B\cos(t)-C\sin(t)-B\cos(t)-C\sin(t)=-2B\cos(t)-2C\sin(t)
\end{gather}

Setting this equal to $\cos(t)$, we find:
\begin{gather}
    -2B=1 \implies B=-\frac{1}{2}\\
    -2C=0 \implies C=0
\end{gather}

Thus, the particular solution for this part is:
\begin{gather}
    x_{p2}(t)=-\frac{1}{2}\cos(t)
\end{gather}

\section{General Solution}
Combining the homogeneous and particular solutions, we have the general solution:
\begin{gather}
    x(t)=C_1 e^{t}+C_2 e^{-t}-\frac{1}{2}t e^{-t}-\frac{1}{2}\cos(t)
\end{gather}

We can then apply the initial conditions to solve for $C_1$ and $C_2$.
Using $x(0)=1$:
\begin{gather}
    1=C_1+C_2-\frac{1}{2}\cos(0)\\
    1=C_1+C_2-\frac{1}{2}\\
    C_1+C_2=\frac{3}{2}
\end{gather}

Using $\dot{x}(0)=0$:
\begin{gather}
    \dot{x}(t)=C_1 e^{t}-C_2 e^{-t}-\frac{1}{2}e^{-t}+\frac{1}{2}t e^{-t}+\frac{1}{2}\sin(t)\\
    0=C_1-C_2-\frac{1}{2}
\end{gather}
Solving the system of equations:
\begin{gather}
    C_1+C_2=\frac{3}{2}\\
    C_1-C_2=\frac{1}{2}
\end{gather}

Adding the two equations, we find:
\begin{gather}
    2C_1=2 \implies C_1=1
\end{gather}
Substituting back to find $C_2$:
\begin{gather}
    1+C_2=\frac{3}{2} \implies C_2=\frac{1}{2}
\end{gather}
Thus, the final solution to the differential equation is:
\begin{gather}
    x(t)=e^{t}+\frac{1}{2}e^{-t}-\frac{1}{2}t e^{-t}-\frac{1}{2}\cos(t)
\end{gather}

\end{document}