\documentclass[11pt]{article}
\usepackage[T1]{fontenc}
\usepackage{lmodern}
\usepackage[margin=1in]{geometry}
\usepackage{amsmath,amssymb}
\usepackage{array}
\usepackage{booktabs}

\pagestyle{empty}
\setlength{\parindent}{0pt}
\setlength{\parskip}{8pt}

\begin{document}

\begin{center}
\LARGE{\textbf{Laplace Transform Sample Problem 2}}
\end{center}

\vspace{10pt}

This problem is question 1.2 from the Fall 2023 QE exam:\\

Find the continuous solution of the following integro-differential equation for z(t)
using the properties of Laplace transforms:

\begin{gather}
    z(t) - (1-t)e^t = \int_0^t z(t-\tau)z(\tau) d\tau\\
    z(0) = 0
\end{gather}

\section{Convolution Theorem}

Using the convolution theorem, we can take the Laplace transform of the right side. Here, 
the integral is just the convolution of \( z(t) \) with itself.
\begin{gather}
    \int_0^t z(t-\tau)z(\tau) d\tau = z(t) * z(t)\\
    \mathcal{L}[z(t) * z(t)] = Z(s) \cdot Z(s) = (Z(s))^2
\end{gather}

\section{Laplace Transform of Left Side}

We can first reqrite the left side of the equation:
\begin{equation}
    z(t) - (1-t)e^t = z(t) - e^t + te^t
\end{equation}

Now we can take the Laplace transform of each term:
\begin{gather}
    \mathcal{L}[z(t)] = Z(s) \\
    \mathcal{L}[e^t] = \frac{1}{s-1} \\
    \mathcal{L}[te^t] = \frac{1}{(s-1)^2}
\end{gather}

\section{Combine and Solve for Z(s)}

Putting it all together, we have:
\begin{gather}
    (Z(s))^2 = Z(s) - \frac{1}{s-1} + \frac{1}{(s-1)^2}\\
    (Z(s))^2 = Z(s) - \frac{s-2}{(s-1)^2}\\
    (Z(s))^2 - Z(s) + \frac{s-2}{(s-1)^2} = 0
\end{gather}

Using the quadratic formula to solve for \( Z(s) \):
\begin{equation}
    Z(s) = \frac{1 \pm \sqrt{1 - 4\frac{s-2}{(s-1)^2}}}{2}
\end{equation}

We can then expand the term under the square root.
\begin{gather}
    Z(s) = \frac{1 \pm \sqrt{\frac{(s-1)^2 - 4(s-2)}{(s-1)^2}}}{2}\\
    Z(s) = \frac{1 \pm \frac{\sqrt{s^2 - 2s + 1 - 4s + 8}}{s-1}}{2}\\
    Z(s) = \frac{1 \pm \frac{\sqrt{s^2 - 6s + 9}}{s-1}}{2}\\
    Z(s) = \frac{1 \pm \frac{s-3}{s-1}}{2}
\end{gather}

This gives us two possible solutions for \( Z(s) \):
\begin{gather}
    Z_+(s) = \frac{1 + \frac{s-3}{s-1}}{2} = \frac{s-2}{s-1}\\
    Z_-(s) = \frac{1 - \frac{s-3}{s-1}}{2} = \frac{1}{s-1}
\end{gather}

$Z_+$ is not valid since it is not continous at $t=0$ (its inverse Laplace transform has a 
jump discontinuity). Thus, we take $Z_-$:
\begin{gather}
    Z(s) = \frac{1}{s-1}
\end{gather}

\section{Inverse Laplace Transform}

Taking the inverse Laplace transform gives:
\begin{gather}
    z(t) = \mathcal{L}^{-1}\left[\frac{1}{s-1}\right] = e^t
\end{gather}

Nice.

\end{document}