\documentclass[11pt]{article}
\usepackage[margin=1in]{geometry}
\usepackage{amsmath,amssymb}
\usepackage{array}
\usepackage{booktabs}

\pagestyle{empty}
\setlength{\parindent}{0pt}
\setlength{\parskip}{8pt}

\begin{document}

\begin{center}
\LARGE{\textbf{Laplace Transforms}}
\end{center}

\vspace{10pt}

\section*{Definition}

The Laplace Transform and its inverse:
$\mathcal{L}\{f(t)\} = F(s) = \int_0^\infty e^{-st} f(t) \, dt$

$\mathcal{L}^{-1}\{F(s)\} = f(t)$

\section*{Common Laplace Transforms}

\begin{center}
\begin{tabular}{c c}
\toprule
\textbf{Function } $f(t)$ & \textbf{Transform } $F(s) = \mathcal{L}\{f(t)\}$ \\
\midrule
$1$ & $\displaystyle\frac{1}{s}$ \\[8pt]
$t$ & $\displaystyle\frac{1}{s^2}$ \\[8pt]
$t^n$ & $\displaystyle\frac{n!}{s^{n+1}}$ \\[8pt]
$e^{at}$ & $\displaystyle\frac{1}{s-a}$ \\[8pt]
$t e^{at}$ & $\displaystyle\frac{1}{(s-a)^2}$ \\[8pt]
$t^n e^{at}$ & $\displaystyle\frac{n!}{(s-a)^{n+1}}$ \\[8pt]
$\sin(\omega t)$ & $\displaystyle\frac{\omega}{s^2 + \omega^2}$ \\[8pt]
$\cos(\omega t)$ & $\displaystyle\frac{s}{s^2 + \omega^2}$ \\[8pt]
$e^{at}\sin(\omega t)$ & $\displaystyle\frac{\omega}{(s-a)^2 + \omega^2}$ \\[8pt]
$e^{at}\cos(\omega t)$ & $\displaystyle\frac{s-a}{(s-a)^2 + \omega^2}$ \\[8pt]
$\sinh(\omega t)$ & $\displaystyle\frac{\omega}{s^2 - \omega^2}$ \\[8pt]
$\cosh(\omega t)$ & $\displaystyle\frac{s}{s^2 - \omega^2}$ \\[8pt]
$u(t-a)$ & $\displaystyle\frac{e^{-as}}{s}$ \\[8pt]
$\delta(t)$ & $1$ \\[8pt]
$\delta(t-a)$ & $e^{-as}$ \\[8pt]
\bottomrule
\end{tabular}
\end{center}

\section*{Properties of Laplace Transforms}

\textbf{Linearity:}
$\mathcal{L}\{af(t) + bg(t)\} = aF(s) + bG(s)$

\textbf{First Derivative:}
$\mathcal{L}\{f'(t)\} = sF(s) - f(0)$

\textbf{Second Derivative:}
$\mathcal{L}\{f''(t)\} = s^2F(s) - sf(0) - f'(0)$

\textbf{$n$-th Derivative:}
$\mathcal{L}\{f^{(n)}(t)\} = s^nF(s) - s^{n-1}f(0) - s^{n-2}f'(0) - \cdots - f^{(n-1)}(0)$

\textbf{Integration:}
$\mathcal{L}\left\{\int_0^t f(\tau) \, d\tau\right\} = \frac{F(s)}{s}$

\textbf{Time Shift (Second Shifting Theorem):}
$\mathcal{L}\{f(t-a)u(t-a)\} = e^{-as}F(s), \quad a \geq 0$
$\mathcal{L}\{g(t)u(t-a)\} = e^{-as}\mathcal{L}\{g(t+a)\}$

\textbf{Frequency Shift (s-Shift Theorem):}
$\mathcal{L}\{e^{at}f(t)\} = F(s-a)$

\textbf{Scaling:}
$\mathcal{L}\{f(at)\} = \frac{1}{a}F\left(\frac{s}{a}\right), \quad a > 0$

\textbf{Multiplication by $t$:}
$\mathcal{L}\{t f(t)\} = -F'(s)$
$\mathcal{L}\{t^n f(t)\} = (-1)^n F^{(n)}(s)$

\textbf{Division by $t$:}
$\mathcal{L}\left\{\frac{f(t)}{t}\right\} = \int_s^\infty F(\sigma) \, d\sigma$

\section*{Convolution Theorem}

\textbf{Convolution of two functions:}
$(f * g)(t) = \int_0^t f(\tau)g(t-\tau) \, d\tau = \int_0^t f(t-\tau)g(\tau) \, d\tau$

\textbf{Convolution Theorem:}
$\mathcal{L}\{(f * g)(t)\} = F(s) \cdot G(s)$

\textbf{Inverse form:}
$\mathcal{L}^{-1}\{F(s) \cdot G(s)\} = (f * g)(t)$

\section*{Initial and Final Value Theorems}

\textbf{Initial Value Theorem:}
$\lim_{t \to 0^+} f(t) = \lim_{s \to \infty} sF(s)$
(provided the limit exists)

\textbf{Final Value Theorem:}
$\lim_{t \to \infty} f(t) = \lim_{s \to 0} sF(s)$
(provided the limit exists and $f(t)$ has a final value; all poles of $sF(s)$ must have negative real parts except possibly a simple pole at $s=0$)

\section*{Inverse Laplace Transforms}

\textbf{Partial Fraction Decomposition:}

For a rational function $\displaystyle\frac{P(s)}{Q(s)}$ where $\deg(P) < \deg(Q)$:

\textbf{Case 1: Simple Real Roots}

If $Q(s) = (s-a_1)(s-a_2)\cdots(s-a_n)$ with distinct roots:
$\frac{P(s)}{Q(s)} = \frac{A_1}{s-a_1} + \frac{A_2}{s-a_2} + \cdots + \frac{A_n}{s-a_n}$

\textbf{Case 2: Repeated Real Roots}

If $Q(s)$ has $(s-a)^n$ as a factor:
$\text{Include terms: } \frac{A_1}{s-a} + \frac{A_2}{(s-a)^2} + \cdots + \frac{A_n}{(s-a)^n}$

\textbf{Case 3: Complex Conjugate Roots}

If $Q(s)$ has a factor $(s-\alpha)^2 + \beta^2$:
$\text{Include term: } \frac{As + B}{(s-\alpha)^2 + \beta^2}$

\textbf{Heaviside Cover-Up Method:}

For a simple root at $s = a$, the coefficient $A$ is:
$A = \lim_{s \to a} (s-a)\frac{P(s)}{Q(s)}$

Or equivalently, substitute $s=a$ into $\displaystyle\frac{P(s)}{Q(s)}$ after "covering up" the factor $(s-a)$ in $Q(s)$.

\section*{Solving ODEs with Laplace Transforms}

\textbf{General Procedure:}
\begin{enumerate}
\item Take the Laplace transform of both sides of the ODE
\item Apply the derivative properties using the initial conditions
\item Solve the resulting algebraic equation for $Y(s)$
\item Use partial fractions to decompose $Y(s)$ if necessary
\item Take the inverse Laplace transform to obtain $y(t)$
\end{enumerate}

\textbf{Example for second-order ODE:}

Given: $y'' + ay' + by = f(t)$ with initial conditions $y(0)$ and $y'(0)$

Taking Laplace transforms:
$s^2Y(s) - sy(0) - y'(0) + a[sY(s) - y(0)] + bY(s) = F(s)$

Solving for $Y(s)$:
$Y(s) = \frac{F(s) + sy(0) + y'(0) + ay(0)}{s^2 + as + b}$

\section*{Special Functions}

\textbf{Unit Step Function (Heaviside Function):}
$u(t-a) = \begin{cases} 0 & \text{if } t < a \\ 1 & \text{if } t \geq a \end{cases}$

Used to "turn on" functions at time $t = a$

\textbf{Dirac Delta Function (Unit Impulse):}
$\delta(t-a) = 0 \text{ for all } t \neq a$
$\int_{-\infty}^\infty \delta(t-a) \, dt = 1$

\textbf{Sifting Property:}
$\int_{-\infty}^\infty f(t)\delta(t-a) \, dt = f(a)$

In ODEs, $\delta(t-a)$ represents an instantaneous impulse at $t=a$

\section*{Solving Integral Equations}

For equations of the form:
$y(t) = f(t) + \int_0^t K(t-\tau)y(\tau) \, d\tau$

\textbf{Solution Method:}
\begin{enumerate}
\item Take Laplace transform of both sides
\item Use convolution theorem: $\mathcal{L}\{\text{integral term}\} = K(s)Y(s)$
\item Obtain: $Y(s) = F(s) + K(s)Y(s)$
\item Solve algebraically: $\displaystyle Y(s) = \frac{F(s)}{1 - K(s)}$
\item Take inverse Laplace to find $y(t)$
\end{enumerate}

\section*{Useful Trigonometric Identities}

$\sin^2(\omega t) = \frac{1 - \cos(2\omega t)}{2}$

$\cos^2(\omega t) = \frac{1 + \cos(2\omega t)}{2}$

$\sin(\omega t)\cos(\omega t) = \frac{\sin(2\omega t)}{2}$

$\sin(A \pm B) = \sin A \cos B \pm \cos A \sin B$

$\cos(A \pm B) = \cos A \cos B \mp \sin A \sin B$

\section*{Common Partial Fraction Results}

$\frac{1}{s(s+a)} = \frac{1}{a}\left(\frac{1}{s} - \frac{1}{s+a}\right)$

$\frac{1}{(s+a)(s+b)} = \frac{1}{b-a}\left(\frac{1}{s+a} - \frac{1}{s+b}\right), \quad a \neq b$

$\frac{s}{s^2 + \omega^2} \quad \xrightarrow{\mathcal{L}^{-1}} \quad \cos(\omega t)$

$\frac{\omega}{s^2 + \omega^2} \quad \xrightarrow{\mathcal{L}^{-1}} \quad \sin(\omega t)$

$\frac{1}{(s+a)^2} \quad \xrightarrow{\mathcal{L}^{-1}} \quad te^{-at}$

$\frac{1}{s^2(s+a)} = \frac{1}{a^2}\left(\frac{1}{s} - \frac{1}{s+a}\right) - \frac{1}{a}\left(\frac{1}{s^2}\right)$

\end{document}