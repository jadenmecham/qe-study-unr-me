\documentclass[11pt]{article}
\usepackage[T1]{fontenc}
\usepackage{lmodern}
\usepackage[margin=1in]{geometry}
\usepackage{amsmath,amssymb}
\usepackage{array}
\usepackage{booktabs}

\pagestyle{empty}
\setlength{\parindent}{0pt}
\setlength{\parskip}{8pt}

\begin{document}

\begin{center}
\LARGE{\textbf{Laplace Transform Sample Problem 1}}
\end{center}

\vspace{10pt}

This problem is question 1.1 from the Fall 2023 QE exam:\\

Using Laplace methods, solve the following differntial equation.

\begin{gather}
    y'''(t)+2y''(t)+y'(t)=\delta(t) \\ 
    y(0)= 0, \quad y'(0) = 0
\end{gather}

\section{Take the Laplace transform of both sides}

\begin{equation}
    \mathcal{L}[y'''(t)+2y''(t)+y'(t)] = \mathcal{L}[\delta(t)]
\end{equation}

Using the linearity property of the Laplace transform, we can separate the left side.
\begin{equation}
    \mathcal{L}[y'''(t)] + 2\mathcal{L}[y''(t)] + \mathcal{L}[y'(t)] = \mathcal{L}[\delta(t)]
\end{equation}

Then we can apply the Laplace transform to each term.
\begin{gather}
    \mathcal{L}[y'''(t)] = s^3Y(s) - s^2y(0) - sy'(0) - y''(0) = s^3Y(s) - y''(0) \\
    \mathcal{L}[y''(t)] = s^2Y(s) - sy(0) - y'(0) = s^2Y(s) \\
    \mathcal{L}[y'(t)] = sY(s) - y(0) = sY(s) \\
\end{gather}

Now substituting these into the equation and cancelling out zeroes from intial conditions gives:
\begin{equation}
    s^3Y(s) - y''(0) + 2s^2Y(s) + sY(s) = \mathcal{L}[\delta(t)]
\end{equation}

The Laplace transform of the delta function is 1, so we have:
\begin{equation}
    s^3Y(s) + 2s^2Y(s) + sY(s) - y''(0) = 1
\end{equation}

\section{Solve for Y(s)}

Factoring out \( Y(s) \) on the left side:

\begin{equation}
    Y(s)(s^3 + 2s^2 + s) - y''(0) = 1
\end{equation}

Rearranging to solve for \( Y(s) \):
\begin{equation}
    Y(s) = \frac{1 + y''(0)}{s^3 + 2s^2 + s}
\end{equation}

Here, $y''(0) = 0$ since the jump in $\delta$ happens at $t=0$ and the function is continuous. 
This leaves us with:

\begin{gather}
    Y(s) = \frac{1}{s^3 + 2s^2 + s}\\
    = \frac{1}{s(s^2 + 2s + 1)} \\
    = \frac{1}{s(s+1)^2}
\end{gather}

\section{Partial fraction decomposition}

We can decompose \( Y(s) \) into partial fractions:
\begin{equation}
    Y(s) = \frac{A}{s} + \frac{B}{s+1} + \frac{C}{(s+1)^2}
\end{equation}

Multiplying both sides by the denominator \( s(s+1)^2 \) gives:
\begin{equation}
    1 = A(s+1)^2 + Bs(s+1) + Cs
\end{equation}

To start solving for the constants, we can choose $s=0$:
\begin{gather}
    1 = A(0+1)^2 + B(0)(0+1) + C(0) \\
    A = 1
\end{gather}

Now, we can choose $s=-1$:
\begin{gather}
    1 = A(-1+1)^2 + B(-1)(-1+1) + C(-1) \\
    1 = 0 + 0 - C \\
    C = -1
\end{gather}

Finally, choose $s=1$:
\begin{gather}
    1 = A(1+1)^2 + B(1)(1+1) + C(1) \\
    1 = 4A + 2B + C \\
    1 = 4(1) + 2B - 1 \\
    1 = 3 + 2B \\
    -2 = 2B \\
    B = -1
\end{gather}

Thus, we have:
\begin{equation}
    Y(s) = \frac{1}{s} - \frac{1}{s+1} - \frac{1}{(s+1)^2}
\end{equation}

\section{Take the inverse Laplace transform}

Now we can take the inverse Laplace transform to find \( y(t) \):
\begin{equation}
    y(t) = \mathcal{L}^{-1}\left[\frac{1}{s}\right] - \mathcal{L}^{-1}\left[\frac{1}{s+1}\right] - \mathcal{L}^{-1}\left[\frac{1}{(s+1)^2}\right]
\end{equation}

Using known inverse Laplace transforms:
\begin{gather}
    \mathcal{L}^{-1}\left[\frac{1}{s}\right] = 1 \\
    \mathcal{L}^{-1}\left[\frac{1}{s+1}\right] = e^{-t} \\
    \mathcal{L}^{-1}\left[\frac{1}{(s+1)^2}\right] = te^{-t}
\end{gather}

Substituting these back in gives:
\begin{gather}
    y(t) = 1 - e^{-t} - te^{-t}\\
    \boxed{y(t) = 1 - e^{-t}(1 - t)}
\end{gather}

Nice.

\end{document}