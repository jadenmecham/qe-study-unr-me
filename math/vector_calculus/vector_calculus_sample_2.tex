\documentclass[11pt]{article}
\usepackage[T1]{fontenc}
\usepackage{lmodern}
\usepackage[margin=1in]{geometry}
\usepackage{amsmath,amssymb}
\usepackage{array}
\usepackage{booktabs}

\pagestyle{empty}
\setlength{\parindent}{0pt}
\setlength{\parskip}{8pt}

\begin{document}

\begin{center}
\LARGE{\textbf{Vector Calculus Sample Problem 2}}
\end{center}

\vspace{10pt}

This problem is question 5.2 on the Fall 2023 QE.\\

Consider the curve $\mathcal{C}$ parametized as:
\begin{equation}
    \begin{cases}
        x=-1+\cos(2t)\\
        y=\sin(t)\\
        z=pt
    \end{cases}
\end{equation}    

For $t \in (0, 2\pi)$ with $p>0$. Denote with $A$ the point on the helix a $t=0$ and with
$B$ the point at $t=2\pi$. Calculate the work done by the force 
$\vec{F}=yz\hat{x}+xz\hat{y}+xy\hat{z}$ along the curve $\mathcal{C}$ from point
$A$ to point $B$ as a function of the parameter $p$. 

\section{Important Equations}
The work done by a force $\vec{F}$ along a curve $\mathcal{C}$ is given by:
\begin{equation}
    W=\int_{\mathcal{C}} F \cdot d\vec{r}
\end{equation}

\section{Find Points A and B}
TO find the coordinates of points $A$ and $B$, we can substitute their corresponding
values of t into the equation for $\mathcal{C}$. At $t=0$, we have:
\begin{equation}
    A=(-1+\cos(0), \sin(0), p\cdot0)=(0, 0, 0)
\end{equation}

At $t=2\pi$, we have:
\begin{equation}
    B=(-1+\cos(4\pi), \sin(2\pi), p\cdot2\pi)=(0, 0, 2\pi p)
\end{equation}

\section{Force along the Curve}

To calculate the force along the curve, we substitute the parametric equations
into the force equation:
\begin{gather}
    F_x = yz = \sin(t) \cdot pt = pt\sin(t)\\
    F_y = xz = (-1+\cos(2t)) \cdot pt = pt(-1+\cos(2t))\\
    F_z = xy = (-1+\cos(2t)) \cdot \sin(t) = (-1+\cos(2t))\sin(t)
\end{gather}

\section{Find $d\vec{r}$}

To find $d\vec{r}$, we differentiate the parametric equations of the curve with the following
formula:
\begin{gather}
    d\vec{r} = \frac{d\vec{r}}{dt} dt\\
    \frac{d\vec{r}}{dt} = \left(\frac{dx}{dt}\hat{x} + \frac{dy}{dt}\hat{y}+ 
    \frac{dz}{dt}\hat{z}\right)dt
\end{gather}

Differentiating each component, we have:
\begin{gather}
    \frac{dx}{dt} = \frac{d}{dt} [-1+\cos(2t)] =-2\sin(2t)\\
    \frac{dy}{dt} = \frac{d}{dt} [\sin(t)] = \cos(t)\\
    \frac{dz}{dt} = \frac{d}{dt} [pt] = p
\end{gather}

\section{Compute the Work Integral}

Now we can compute the work integral. First, we need to find $F \cdot d\vec{r}$:
\begin{gather}
    F \cdot d\vec{r} = \left(pt\sin(t)(-2\sin(2t)) + pt(-1+\cos(2t))\cos(t) +
    (-1+\cos(2t))\sin(t)p\right) dt
\end{gather}

Thus, the work done by the force along the curve from point A to point B is given by:
\begin{equation}
    W = \int_0^{2\pi} \left(pt\sin(t)(-2\sin(2t)) + pt(-1+\cos(2t))\cos(t) +
    (-1+\cos(2t))\sin(t)p\right) dt
\end{equation}

Evaluating this integral, we have:
\begin{equation}
    W = p \int_0^{2\pi} \left(-2t\sin(t)\sin(2t) + t(-1+\cos(2t))\cos(t) +
    (-1+\cos(2t))\sin(t)\right) dt
\end{equation}

To solve this integral, we can split it into three separate integrals:
\begin{equation}
    W = p \left( \int_0^{2\pi} -2t\sin(t)\sin(2t) dt + \int_0^{2\pi} t(-1+\cos(2t))\cos(t) dt +
    \int_0^{2\pi} (-1+\cos(2t))\sin(t) dt \right)
\end{equation}

Calculating each integral separately, we have:
\begin{gather}
    \int_0^{2\pi} -2t\sin(t)\sin(2t) dt = \frac{8\pi}{3}\\
    \int_0^{2\pi} t(-1+\cos(2t))\cos(t) dt = -\frac{4\pi}{3}\\
    \int_0^{2\pi} (-1+\cos(2t))\sin(t) dt = 0
\end{gather}

Combining these results, we find:
\begin{equation}
    W = p \left( \frac{8\pi}{3} - \frac{4\pi}{3} + 0 \right) = \boxed{\frac{4\pi p}{3}}
\end{equation}

Boom.

\end{document}